\topmargin0pt
\headheight0pt
\headsep0pt
\textheight9in
\oddsidemargin0pt
\textwidth6in
\setlength{\parindent}{0pt}
\setlength{\parskip}{4pt plus 2pt minus 1pt}
\setlist[itemize]{leftmargin=*}
\setlist[enumerate]{leftmargin=*}

% \task[note]{title}{text}
\newcommand{\task}[3][]{
	\item \textbf{\sffamily #2} \hfill {\small \textsc{#1}}
	
	\ifthenelse{\equal{\theexercise}{overview}}{}{#3}
}

\ifsolution
	\newtcolorbox{solution}{%
		title={\setlength{\parskip}{0ex}\vphantom{/}\footnotesize\t{Solution}{Lösung}},%
		colback=white,%
		colframe=red!30,%
		coltitle=black,%
		fonttitle=\bfseries\sffamily,%
		left=1mm,%
		right=1mm,%
		top=1mm,%
		bottom=1mm,%
		breakable,%
		before upper={\sffamily\setlength{\parskip}{2pt}\footnotesize}%
	}
\else
	\newenvironment{solution}{\setbox0=\vbox\bgroup}{\egroup}
\fi

% \deftask{name}{title}{text}
\newcommand{\deftask}[3]{\expandafter\def\csname #1\endcsname[##1]{\task[##1]{#2}{#3}}}

% \exercise{number}{title}{schedule}{content}
\newcommand{\exercise}[4]{
	\ifthenelse{\equal{\theexercise}{#1} \OR \equal{\theexercise}{overview}}{
		\ifthenelse{\equal{\theexercise}{#1}}{
			\ifdefined\theuniversity
				\ifthenelse{\equal{\theuniversity}{paderborn}}{%
					\includegraphics[width=0.2\linewidth]{upb}
					\hfill
					\includegraphics[width=0.16\linewidth]{upb-triangles}\\
				}{}
				\ifthenelse{\equal{\theuniversity}{braunschweig}}{%
					\includegraphics[width=0.25\linewidth]{TUBraunschweig_RGB_beamer}
					\hfill
					\includegraphics[width=0.16\linewidth]{ISF_Logo}\\
				}{}
				\ifthenelse{\equal{\theuniversity}{magdeburg}}{%
					\includegraphics[width=0.26\linewidth]{logos/ovgu-blue}
					\hfill
					\includegraphics[width=0.21\linewidth]{dbse}\\
				}{}
			\fi
			{\color{lightgray}\hrule}\bigskip
			\begin{LARGE}
				\sffamily
				\centerline{\textbf{\thetitle{}}}
			\end{LARGE}
			\medskip
		}{}
		\begin{Large}
			\sffamily
			\ifthenelse{\equal{\theexercise}{#1}}{\centerline}{}{\ifnum0<0#1\relax \t{Exercise}{Übung} #1: \fi #2 \ifthenelse{\equal{\theexercise}{overview}}{}{\ifsolution (\t{Solution Sheet}{Lösungsblatt}) \fi}}
		\end{Large}
		\ifthenelse{\equal{\theexercise}{#1}}{
			\medskip

			\theauthor \ifdefined\thesemester{}\hfill \thesemester{}\fi
			
			\ifdefined\theroom{}\t{Room}{Raum} \theroom{}\fi \ifthenelse{\equal{#3}{}}{}{\hfill #3}
			
			\vspace*{1.5ex}%
			{\color{lightgray}\hrule\medskip}
		}{}
		\begin{enumerate}
			#4
		\end{enumerate}
	}{}
}

\tcbuselibrary{breakable}

\newcommand{\mydefinition}[2]{
	{\sffamily\begin{tcolorbox}[title={\setlength{\parskip}{0ex}\vphantom{/}#1},colback=white,colframe=orange!30,coltitle=black,fonttitle=\bfseries,left=1mm,right=1mm,top=1mm,bottom=1mm,breakable]
		#2
	\end{tcolorbox}}
}

\newcommand{\myexample}[2]{
	{\sffamily
	\begin{tcolorbox}[title={\setlength{\parskip}{0ex}\vphantom{/}#1},colback=white,colframe=blue!30,coltitle=black,fonttitle=\bfseries,left=1mm,right=1mm,top=1mm,bottom=1mm,breakable]
		#2
	\end{tcolorbox}}
}

\newcommand{\mynote}[2]{
	{\sffamily
	\begin{tcolorbox}[title={\setlength{\parskip}{0ex}\vphantom{/}#1},colback=white,colframe=red!30,coltitle=black,fonttitle=\bfseries,left=1mm,right=1mm,top=1mm,bottom=1mm,breakable]
		#2
	\end{tcolorbox}}
}

\input{../Course-on-Software-Product-Lines/slides/template/colorscheme}
\input{../Course-on-Software-Product-Lines/slides/template/macros}
\input{../Course-on-Software-Product-Lines/slides/template/featurediagrams}

\makeatletter
\let\@noitemerr\relax
\makeatother

\graphicspath{{../Course-on-Software-Product-Lines/pics/}{../pics/}{../pics/papers/}{../pics/code/}{../pics/logos/}}

\renewcommand{\emph}[1]{{\textit{#1}}}

\newcommand{\explanationItem}[1]{\item[] \textcolor{gray}{\footnotesize #1}}
\newcommand{\commandItem}[1]{\item[>] \texttt{#1}}