\deftask{CloneAndOwnFundamentals}
{\t{Clone-and-Own Fundamentals}{Grundlagen zu Clone-and-Own}}{
	\begin{enumerate}
		\item \t{
			What is clone-and-own and what are its use-cases?
		}{
			Was ist Clone-and-Own?
			Was sind Anwendungsfälle für Clone-and-Own?
		}
		\begin{solution}
REDACTED
		\end{solution}
		
		\item \t{
			A team develops a mobile app to manage school timetables.
			One school wants to remove weekend classes.
			Another school adds support for rotating weekly schedules.
			A third wants to show exam dates in a special tab.
			How could the Clone-and-Own approach be used here?
		}{
			Ein Team entwickelt eine mobile App zur Verwaltung von Stundenplänen.
			Eine Schule möchte keine Unterrichtsstunden am Wochenende haben.
			Eine andere Schule möchte rotierende Wochenpläne unterstützen.
			Eine dritte Schule möchte Prüfungs\-ter\-mine in einem speziellen Tab anzeigen.
			Wie könnte der Clone-and-Own-Ansatz hier verwendet werden?
		}
		\begin{solution}
REDACTED
		\end{solution}
	\end{enumerate}
}

\deftask{CloneAndOwnStrategicDecisions}
{\t{Strategic Decisions in Clone-and-Own}{Strategische Entscheidungen bei Clone-and-Own}}{
	\t{
		A startup is building a fitness app.
		The first version tracks steps and calories.
		Later, a developer clones it to create an athlete-focused variant with features like heart rate monitoring and sleep tracking.
		Another team then clones the first variant to add smartwatch support.
		Eventually, the company plans to release a premium app that includes all features across variants — but the teams now face unexpected challenges.
	}{
		Ein Startup entwickelt eine Fitness-App.
		Die erste Version verfolgt Schritte und Kalorien.
		Später klont ein Entwickler sie, um eine sportlerorientierte Variante mit Funktionen wie Herzfrequenzüberwachung und Schlafüberwachung zu erstellen.
		Ein anderes Team klont dann die erste Variante, um Smartwatch-Unterstützung hinzuzufügen.
		Schließlich plant das Unternehmen, eine Premium-App zu veröffentlichen, die alle Funktionen der Varianten enthält — aber die Teams stehen nun vor unerwarteten Herausforderungen.
	}
	\begin{enumerate}
		\item \t{
			Why might the developers have initially chosen the Clone-and-Own approach, even knowing its limitations?
		}{
			Warum haben die Entwickler möglicherweise zunächst den Clone-and-Own-Ansatz gewählt, obwohl sie dessen Grenzen kannten?
		}
		\item \t{
			At what point does this strategy start to create problems in the scenario above? Which problems could arise?
		}{
			An welchem Punkt beginnt diese Strategie im obigen Szenario Probleme zu verursachen? Welche Probleme könnten auftreten?
		}
		\item \t{
			Would you have recommended Clone-and-Own for rapid feature prototyping if you were leading this startup in its early phase?
			Justify your answer.
		}{
			Hättest du Clone-and-Own für schnelles Feature-Prototyping empfohlen, wenn du dieses Startup in seiner frühen Phase geleitet hättest?
			Begründe deine Antwort.
		}

		\begin{solution}
REDACTED
		\end{solution}
	\end{enumerate}
}

\deftask{CloneAndOwnVersionControlSystems}
{\t{%
	Managing Variants with Version Control Systems
}{%
	Variantenverwaltung mit Versionsverwaltungssystemen
}}{
	\t{
		A company creates an e-learning platform with different variants for schools, universities, and corporate training by cloning the main system.
		The company is currently using a version control system to manage the variants.
	}{
		Ein Unternehmen erstellt eine E-Learning-Plattform mit verschiedenen Varianten für Schulen, Universitäten und Unternehmen, indem das Hauptsystem geklont wird.
		Das Unternehmen nutzt derzeit ein Versionsverwaltungssystem, um die Varianten zu verwalten.
	}

	\begin{enumerate}
		\item \t{
			Explain and differentiate the terms version, revision, and variant using the example above.
		}{
			Erklären und unterscheiden Sie die Begriffe Version, Revision und Variante anhand dieses Beispiels.
		}
		\begin{solution}
REDACTED
		\end{solution}

		\item \t{
			How can we use version control systems to manage these different variants?
		}{
			Wie können Versionsverwaltungssysteme genutzt werden, um diese verschiedenen Varianten zu verwalten?
		}
		\begin{solution}
REDACTED
		\end{solution}
	\end{enumerate}
}

\deftask{CloneAndOwnBuildSystems}
{\t{Managing Variants with Build Systems}{Variantenverwaltung mit Buildsystemen}}{
	\t{
		Consider a company developing a mobile game with different variants for casual gamers, competitive players, and premium users.
	}{
		Es geht um ein Unternehmen, das ein mobiles Spiel mit verschiedenen Varianten für Gelegenheitsspieler, Wettkampfspieler und Premium-Nutzer entwickelt.
	}
	\begin{enumerate}
		\item \t{
			What is a build system?
		}{
			Was ist ein Build-System?
		}
		\begin{solution}
REDACTED
		\end{solution}
		\item \t{
			In the example above, which tasks of the development process can be realized via a build system?
		}{
			Bezogen auf das obige Beispiel, welche Aufgaben im Software-Entwicklungsprozess können durch ein Build-System umgesetzt werden?
		}
		\begin{solution}
REDACTED
		\end{solution}

		\item \t{
			What steps are needed to manage and configure variants with the build systems?
		}{
			Welche Schritte sind erforderlich, um Varianten mit Build-Systemen zu verwalten und zu konfigurieren?
		}
		\begin{solution}
REDACTED
		\end{solution}
	\end{enumerate}
}

\deftask{CloneAndOwnComparison}
{\t{Version Control Systems vs. Runtime Variability}{Versionsverwaltungssysteme oder Laufzeitvariabilität?}}{
	\begin{enumerate}
		\item \t{
			Compare the advantages and disadvantages of using a version-control-based approach versus implementing a product line with runtime variability.
		}{
			Vergleiche die Vor- und Nachteile eines versionskontrollbasierten Ansatzes mit der Implementierung einer Produktlinie mit Laufzeitvariabilität.
		}
		\begin{solution}
REDACTED
		\end{solution}

		\item \t{
			Compare the three different strategies for clone-and-own as discussed in the lecture.
		}{
			Vergleiche die drei verschiedenen Strategien für Clone-and-Own, wie sie in der Vorlesung besprochen wurden.
		}
		\begin{solution}
REDACTED
		\end{solution}
	\end{enumerate}
}