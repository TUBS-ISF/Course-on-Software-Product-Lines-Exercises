\deftask{ConditionalPreprocessor}
{\t{Preprocessor-Based Variability}{Präprozessor-basierte Variabilität}}{
	\t{
		You have inherited a codebase, in which the previous developers heavily used preprocessor macros.
		Below is a snippet from the thermal monitoring module.
	}{
		Du hast eine Codebasis geerbt, in der die vorherigen Entwickler intensiv Präprozessor-Makros verwendet haben.
		Unten ist ein Ausschnitt aus dem Temperaturüberwachungs\-modul.
	}
	
	\myexample{\t{C Code Using Preprocessor Directives}{C-Code mit Präprozessor-Direktiven}}{
		\includegraphics[scale=1.1]{code-preprocessor}
	}
	
	\begin{enumerate}
		\item \t{
			What is a preprocessor and how does it support implementing variability in software product lines?
		}{
			Was ist ein Präprozessor und wie unterstützt er die Implementierung von Variabilität in Software-Produktlinien?
		}
		\begin{solution}
REDACTED
		\end{solution}
		\item \t{
			Explain the role of the preprocessor in this code.
		}{
			Erkläre die Rolle des Präprozessors in diesem Code.
		}
		\begin{solution}
REDACTED
		\end{solution}
		
		\item \t{
			Assuming \texttt{PRECISION\_HIGH}, \texttt{UNIT\_KELVIN}, \texttt{MODE\_THERMAL}, and \texttt{SENSOR\_ADVANCED} are defined, while \texttt{EPSILON}, \texttt{SENSOR\_COUNT}, \texttt{CONVERT}, and \texttt{p\_t} are not pre-defined outside the given code snippet.
			Show the resulting source code as it would appear after preprocessing, with all macros expanded and conditional compilation resolved.
		}{
			Angenommen, \texttt{PRECISION\_HIGH}, \texttt{UNIT\_KELVIN}, \texttt{MODE\_THERMAL} und \texttt{SENSOR\_ADVANCED} sind definiert, während \texttt{EPSILON}, \texttt{SENSOR\_COUNT}, \texttt{CONVERT} und \texttt{p\_t} außerhalb des gegebenen Code-Snippets nicht vordefiniert sind.
			Gib den resultierenden Quellcode an, wie er nach dem Preprocessing erscheinen würde, mit allen expandierten Makros und aufgelöster bedingter Kompilierung.
		}
		\begin{solution}
REDACTED
		\end{solution}

		\item \t{
			Here we used the C preprocessor (CPP), but not all preprocessors are equally suitable in all contexts.
			Compare the three preprocessors \texttt{CPP}, \texttt{Munge}, and \texttt{Antenna}.
			In which types of projects would each be most appropriate?
		}{
			Hier wurde der C-Präprozessor (CPP) verwendet, aber nicht alle Präprozessoren sind in allen Kontexten gleichermaßen geeignet.
			Vergleiche die drei Präprozessoren \texttt{CPP}, \texttt{Munge} und \texttt{Antenna}.
			Für welche Arten von Projekten wäre jeder am besten geeignet?
		}
		\begin{solution}
REDACTED
		\end{solution}
	\end{enumerate}
}

\deftask{ConditionalLinuxKernel}
{\t{Variability in the Linux Kernel}{Variabilität im Linux-Kernel}}{
	\begin{enumerate}
		\item \t{
			Explain the role of \textsc{KConfig}, \textsc{KBuild}, \textsc{CPP}, and \textsc{MenuConfig}.
			How do they differ, and what is the purpose of each?
		}{
			Erkläre die Rolle von \textsc{KConfig}, \textsc{KBuild}, \textsc{CPP} und \textsc{MenuConfig}.
			Wie unterscheiden sie sich und was ist jeweils ihr Zweck?
		}
		\begin{solution}
REDACTED
		\end{solution}
		
		\item \t{
			The Linux kernel uses a combination of tools to manage variability.
			Below is a simplified \textsc{KBuild} snippet:
		}{
			Der Linux-Kernel verwendet eine Kombination von Tools zur Verwaltung von Variabilität.
			Unten ist ein vereinfachtes \textsc{KBuild}-Snippet:
		}
		\myexample{\t{KBuild Snippet from Kernel Makefile}{KBuild-Snippet aus Kernel-Makefile}}{
			\includegraphics[scale=1.1]{code-kbuild-makefile}
		}
		
		\begin{enumerate}
			\item \t{
				Explain what each line in the Makefile snippet does.
				How do \texttt{obj-y}, \texttt{obj-m}, and \texttt{obj-\$(...)} control the inclusion of features in the build?
			}{
				Erkläre, was jede Zeile im Makefile-Snippet macht.
				Wie steuern \texttt{obj-y}, \texttt{obj-m} und \texttt{obj-\$(...)} die Einbindung von Features im Build?
			}
			\begin{solution}
REDACTED
			\end{solution}
			
			\item \t{
				Given the following configuration options:
				\begin{itemize}
					\item \texttt{CONFIG\_DEBUG\_MODE=y}
				\end{itemize}
				Which directories or files are compiled statically, treated as modules, or excluded from the build?
			}{
				Gegeben seien die folgenden Konfigurationsoptionen:
				\begin{itemize}
					\item \texttt{CONFIG\_DEBUG\_MODE=y}
				\end{itemize}
				Welche Verzeichnisse oder Dateien werden statisch kompiliert, als Module behandelt oder vom Build ausgeschlossen?
			}
			\begin{solution}
REDACTED
			\end{solution}

			\item \t{
				What is the difference between the implementation of features with build systems (like \textsc{KBuild}) and clone-and-own with build systems (as discussed in an earlier exercise)?
			}{
				Was ist der Unterschied zwischen der Implementierung von Features mit Build-Systemen (wie \textsc{KBuild}) und Clone-and-Own mit Build-Systemen (wie in einer früheren Übung besprochen)?
			}
			\begin{solution}
REDACTED
			\end{solution}
		\end{enumerate}
	\end{enumerate}
}

\deftask{ConditionalFeatureTraceability}
{\t{Feature Traceability}{Feature Traceability}}{
	\t{
		You are working on a Smart Grid Controller that adjusts its behavior depending on feature flags related to precision, sensor type, and logging.
		Below is a simplified and configurable code snippet using conditional compilation.
	}{
		Du arbeitest an einem Smart-Grid-Controller, der sein Verhalten abhängig von Feature-Flags für Präzision, Sensortyp und Logging anpasst.
		Unten ist ein vereinfachtes und konfigurierbares Code-Snippet mit bedingter Kompilierung.
	}

	\myexample{\t{Java Code with Preprocessor Directives}{Java-Code mit Präprozessor-Direktiven}}{
		\includegraphics[scale=1.1]{code-gridcontroller-nested}
	}

	\begin{enumerate}
		\item \t{
		 	What is feature traceability?
			Explain challenges related to feature traceability (i.e., code tangling, scattering, and replication).
			To which degree do these apply to this code snippet?
		}{
			Was ist Feature Traceability (Rück\-verfolgbarkeit)?
			Erkläre Herausforderungen im Zusammenhang mit Feature Traceability (Tangling, Scattering, und Replikation).
			Inwieweit treffen diese auf dieses Code-Beispiel zu?
		}
		\begin{solution}
REDACTED
		\end{solution}

		\item \t{
			What are the consequences of code tangling, scattering and replication?
		}{
			Welche Konsequenzen haben Tangling, Scattering und Replikation?
		}
		\begin{solution}
REDACTED
		\end{solution}

		\item \t{
			By which means can feature traceability be ensured?
			Discuss their benefits and drawbacks!
		}{
			Welche Möglichkeiten gibt es, um Feature Traceability sicherzustellen?
			Diskutiere Vor- und Nachteile.
		}
		\begin{solution}
REDACTED
		\end{solution}
	\end{enumerate}
}
	
\deftask{ConditionalComparison}
{\t{%
	Comparison of Variability Implementation Techniques
}{%
	Vergleich von Implementierungstechniken für Variabilität
}}{
	\t{
		Compare the advantages, disadvantages, and use cases of features with preprocessors and features with build systems.
		Also include the techniques discussed in the lecture so far in your comparison.
		Use examples to support your arguments.
	}{
		Vergleiche die Vor- und Nachteile sowie Anwendungsfälle von Features mit Präprozessoren und Features mit Build-Systemen.
		Berücksichtige auch die bisher in der Vorlesung besprochenen Techniken in deinem Vergleich.
		Verwende Beispiele zur Unterstützung deiner Argumente.
	}

	\begin{solution}
REDACTED
	\end{solution}
}