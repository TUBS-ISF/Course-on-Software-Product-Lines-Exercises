\deftask{TestingCombinatorialInteractionTesting}
{\t{Combinatorial Interaction Testing}{Combinatorial Interaction Testing}}{
	\begin{enumerate}
		\item \t{
			How can pairwise interaction sampling be used to detect feature interactions in a software product line?
			What are the differences between pairwise and $t$-wise interaction sampling for $t > 2$?
			What is the difference between $t$-wise interaction coverage and statement, branch, or condition coverage?
		}{
			Wie kann \emph{Pairwise Interaction Sampling} eingesetzt werden, um Feature-Interak\-tio\-nen in einer Produktlinie zu erkennen?
			Wie unterscheiden sich \emph{Pairwise} und \emph{$t$-Wise Interaction Sampling} für $t > 2$?
			Was ist der Unterschied zwischen \emph{$t$-Wise Interaction Coverage} und \emph{Statement-, Branch- oder Condition Coverage}?
		}
		\begin{solution}
REDACTED
		\end{solution}

		\item \t{
			Determine all pairs of features ($t=2$) relevant for sampling the following feature model.
			Which features and which pairs of features do not have to be considered?
		}{
			Bestimme alle für Sampling kombinatorisch relevanten Feature-Paare ($t=2$) für das folgende Feature-Modell.
			Welche Features und Feature-Kombinationen müssen hierbei nicht beachtet werden und warum?
		}
		\myexample{\t{%
			Feature Model
		}{%
			Feature-Modell
		}}{
			\centering\sffamily
			\featureDiagram{
				GraphLibrary,abstract
				[Edge,mandatory,abstract[Directed,concrete,alternative][Undirected,concrete]]
				[Weighted,optional,concrete]
				[Algorithm,optional,abstract[Cycle,concrete,or][ShortestPath,concrete]]
			}
			\\
			$\text{Cycle} \pimplies \text{Directed}$ \\
			$\text{ShortestPath} \pimplies \text{Directed} \pand \text{Weighted}$
		}
		\begin{solution}
REDACTED
		\end{solution}

		\item \t{
			Do the following configurations suffice to cover all relevant pairwise feature interactions from above?
			If they do not, add further configurations until all relevant pairwise interactions are covered.
			(For brevity, we abbreviate feature names to their first letter.)
		}{
			Reichen die folgenden Konfigurationen aus, um alle oben bestimmten paarweisen Feature-Interaktionen abzudecken?
			Wenn nicht, füge weitere Konfigurationen hinzu, bis alle paarweisen Interaktionen abgedeckt sind.
			(Wir kürzen hier die Feature-Namen auf ihren Anfangsbuchstaben ab.)
		}
		\begin{center}
			\begin{tabular}{lcccccccc}
				\hline
				\textbf{Name} & \textbf{G} & \textbf{E} & \textbf{D} & \textbf{U} & \textbf{W} & \textbf{A} & \textbf{C} & \textbf{S} \\ \hline\hline
				$c_1$ & 1 & 1 & 1 & 0 & 0 & 0 & 0& 0 \\ \hline
				$c_2$ & 1 & 1 & 0 & 1 & 0 & 0 & 0& 0 \\ \hline
				$c_3$ & 1 & 1 & 0 & 1 & 1 & 0 & 0& 0 \\ \hline
				$c_4$ & 1 & 1 & 1 & 0 & 1 & 1 & 1& 1 \\ \hline
			\end{tabular}
		\end{center}
		\begin{solution}
REDACTED
		\end{solution}
	\end{enumerate}
}

\deftask{TestingSolutionSpaceSampling}
{\t{Solution-Space Sampling}{Solution-Space Sampling}}{
	\t{
		Consider the following product line for a configurable \texttt{Edge} datatype:
	}{
		Gegeben sei folgende Produktlinie für eine konfigurierbare \texttt{Edge}-Datenstruktur:
	}
	\myexample{\t{Feature Model}{Feature-Modell}}{
		\centering\sffamily
		\featureDiagram{
			Edge,abstract
			[Weighted,optional,concrete]
			[Coloring,optional,abstract
			[SimpleColor,alternative,concrete]
			[MultiColor,concrete]]
			[Direction,mandatory,abstract
			[Directed,alternative,concrete]
			[Undirected,concrete]]
		}\\[1ex]
		$\text{MultiColor} \pimplies \text{Weighted}$
	}
	\myexample{\t{Annotated Source Code}{Annotierter Quelltext}}{
		\begin{tabular}[t]{@{}c@{\hspace{0.02\linewidth}}c@{}}
			\includegraphics[width=0.38\linewidth]{edge-feature-code-1} &
			\raisebox{0ex}{\includegraphics[width=0.42\linewidth]{edge-feature-code-2}}
		\end{tabular}
	}
	\begin{enumerate}
		\item \t{
			Based on the above feature model and source code, construct a minimal sample of valid configurations that achieves coverage of \texttt{\#if} blocks (Tartler et al. 2012).
			Justify your answer.
		}{
			Basiert auf dem obigen Feature-Modell und dem Quelltext, konstruiere ein minimales Sample (also eine Stichprobe) von validen Konfigurationen, das eine Coverage aller \texttt{\#if}-Blöcke (Tartler et al. 2012) erreicht.
			Begründe deine Antwort.
		}
		\begin{solution}
REDACTED
		\end{solution}
		
		\item \t{
			Construct a second minimal sample that achieves pairwise presence-condition coverage (Krieter et al. 2022).
			Justify your answer.
		}{
			Konstruiere ein zweites minimales Sample, das eine \emph{Pairwise Presence-Condition Coverage} (Krieter et al. 2022) erreicht.
			Begründe deine Antwort.
		}
		\begin{solution}
REDACTED
		\end{solution}
		
		\item \t{
			For each of the following strategies, how many configurations do we have to test at least?
			\begin{enumerate}
				\item Product-based strategy
				\item Sample-based strategy using \texttt{\#if} block coverage
				\item Sample-based strategy using pairwise presence-condition coverage
			\end{enumerate}
			Under which circumstances would each strategy be preferred?
		}{
			Wie viele Konfigurationen müssen für jede der folgenden Strategien mindestens getestet werden?
			\begin{enumerate}
				\item Produktbasierte Strategie
				\item Sample-basierte Strategie mit \texttt{\#if}-Block Coverage
				\item Sample-basierte Strategie mit Pairwise Presence-Condition Coverage
			\end{enumerate}
		}
		\begin{solution}
REDACTED
		\end{solution}
	\end{enumerate}
}

\deftask{TestingSamplingDiscussion}
{\t{Limits of Sample-Based Testing}{Grenzen des Sample-basierten Testens}}{
	\begin{enumerate}
		\item \t{
			What are the advantages and disadvantages of relying on a small sample of configurations instead of testing all valid products in a software product line?
		}{
			Was sind die Vor- und Nachteile, wenn man sich auf ein kleines Sample von Konfigurationen verlässt, anstatt alle validen Produkte in einer Software-Produktlinie zu testen?
		}
		\begin{solution}
REDACTED
		\end{solution}
		
		\item \t{
			Besides sample-based testing (covering $t$-wise interactions, \texttt{\#if} blocks, or presence-conditions) and product-based testing, which alternative testing strategies can you think of?
			When would you use which testing strategy?
		}{
			Welche alternativen Teststrategien kannst du dir neben Sample-basiertem Testen (durch Coverage von $t$-wise-Interaktionen, \texttt{\#if}-Blöcken oder Presence Conditions) und produktbasiertem Testen noch vorstellen?
			Wann würdest du welche Teststrategie einsetzen?
		}
		\begin{solution}
REDACTED
		\end{solution}

		\item \t{
			In practice, how can a suitable value of $t$ for $t$-wise interaction sampling be chosen?
			Discuss with examples or scenarios where lower or higher values of $t$ might be appropriate.
		}{
			Wie kann in der Praxis ein geeigneter Wert von $t$ für $t$-Wise Interaction Sampling gewählt werden?
			Diskutiere mit Beispielen oder Szenarien, wo niedrigere oder höhere Werte von $t$ angemessen sein könnten.
		}
		\begin{solution}
REDACTED
		\end{solution}

		\item \t{
			What is the difference between $t$-wise interaction coverage and $t$-wise presence-condition coverage?
			Discuss when which strategy is more appropriate.
		}{
			Was ist der Unterschied zwischen $t$-wise Interaction Coverage und $t$-wise Presence Condition Coverage?
			Diskutiere, welche Strategie wann angemessener ist.
		}
		\begin{solution}
REDACTED
		\end{solution}
	\end{enumerate}
	\begin{solution}
REDACTED
	\end{solution}
}