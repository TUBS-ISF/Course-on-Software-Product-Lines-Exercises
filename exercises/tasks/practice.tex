% Practical tasks in use at TU Braunschweig

\deftask{PracticeAccessCalculatorGit}
{Getting Access to the Repository}{
\textbf{Important:} This requires manual processing by our side.
Please reach out early to get access on time.

Before you can start to develop a new feature, you need access to your own repository.
To get access, please login to the GITZ Gitlab\footnote{\url{https://git.rz.tu-bs.de}} once using GITZ account.
Then, write an email to Marco (\href{mailto:REDACTED?subject=[SPL] Access to Calculator Repository}{REDACTED} \textit{Subject:}[SPL] Access to Calculator Repository) with your official \textbf{TU Braunschweig email} address, containing your
\begin{itemize}
\item \textbf{full name},
\item Gitlab \textbf{account name},
\item \textbf{registration number} (Matrikelnummer), and
\item \textbf{course of study} (Studiengang).
\end{itemize} 
We will then send you back an email with the URL to your own repository.
}

\deftask{PracticeSetupCalculatorGit}
{Getting Started}{
1  For this and following tasks, you need Java\footnote{\url{https://openjdk.org/}} (17+) and Maven\footnote{\url{https://maven.apache.org/}}.
  While this task can be completed with any IDE or text editor, we advise you to use Eclipse\footnote{\url{https://www.eclipse.org/}} because future tasks require Eclipse.
  You can find an Eclipse version with Java development support for Linux, Windows, or Mac here.\footnote{\url{https://www.eclipse.org/downloads/packages/}}

   Please clone your repository from the URL provided by us (see Task 1) using a git client of your choice\footnote{
     In case you are unfamiliar with Git, we recommend you to get familiar with the basics of Git.
   	There are numerous tutorials, guides, and talks online in text as well as video format.
     Git is a version control system with many helpful features next to serving as a plain submission system for this course.
   	To install Git for your system, you can follow the official instructions: \url{https://git-scm.com/downloads}.
     Windows users should also install Git-Bash and integration for external programs.
   	In your terminal (or Git-Bash on Windows) you can check whether you have installed Git correctly via \texttt{git -{}-version}.
     }.
     Using a terminal, you can navigate to a directory of your choice and execute the following command
       \begin{itemize}\small
         \commandItem{git clone <Your Git URL>}
         \explanationItem{%
     		  This copies the content of the Git repository to your local hard drive.
     		}
     		\end{itemize}
 	After cloning open the \texttt{README.md} file.
 	This file contains instructions on how to build and run the project, and how to import the project into Eclipse.
 	Follow these setup steps and run the Calculator.
 	You should see a small window, which implements the calculator.
  % If you are using Eclipse, open Eclipse, then import the Java project by \textit{File} $\rightarrow$ \textit{Import} $\rightarrow$ \textit{General} $\rightarrow$ \textit{Existing Projects into Workspace}.
  % Instructions to build and run the project, as well as more detailed instructions for importing can be found in the \texttt{README.md} of the project.
}

\deftask{PracticeDevelopCalculatorFeature}
{Develop a New Feature}{
	Currently, our calculator is very limited in terms of features.
	We can only perform simple addition and multiplication.
	As the first task, we want you to include an additional feature to the calculator.
	Hereby, a feature could be any functionality you would like to have.
	Some examples are:
	\begin{itemize}
		\item New arithmetic operators (e.g., division, logarithm)
		\item Usability features (e.g., undo)
		\item User interface features (e.g., styling buttons)
	\end{itemize}

	When you are done, \textbf{open the \texttt{README.md} file (e.g., in a text editor) and document} at the bottom of the file which feature you added to the calculator.
	Provide the feature's name and a short description (1-2 sentences).
}

\deftask{PracticeDeleteCalculatorFeature}
{Delete an Existing Feature}{
	\textit{Do I really need this feature?} For this task, we want you to remove a feature.
	To do so, we first need to identify existing features.
	Your task is to identify the features, choose one of them, and remove it from the source code.
	Make sure to fully remove the respective source code (i.e., remove respective code fragments from every affected source file).

	When you are done, \textbf{open the \texttt{README.md} file (e.g., in a text editor) and document} at the bottom of the file which feature you removed from the calculator.
	Provide the feature's name and a short description (1-2 sentences).
}

\deftask{PracticeSubmitCalculator}
{Submit Your Results}{
  Once you finished the previous tasks, submit your adapted calculator by the submission deadline.
%  To submit, we use Git and Gitlab.
	You must push your calculator to your Git repository, which we prepared for you in the GITZ Gitlab\textsuperscript{6}.
%  You will soon receive an e-mail which notifies you that you have been granted access to your repository.
%  Once you got access to your repository, you can push your changes.
You can do this with the git client of your choice.
Using a terminal, you can navigate to the directory containing your calculator and follow these steps:
  \begin{itemize}\small
%    \commandItem{git init -{}-initial-branch=main}
%    \explanationItem{%
%		  This tells Git that you want the current directory to become a repository and that your changes will be tracked in a branch called "main".
%			If you do not yet know what a branch is don't worry, it is not necessary to know for this task.
%		}
    \commandItem{git add .}
    \explanationItem{Stage all current files for submission.}
    \commandItem{git commit -m "<message>"}
    \explanationItem{Commit the changes, i.e., bundle them and treat them as a single change. You must replace \texttt{<message>} by a message of your choice (e.g., \texttt{initial commit}).}
    % \item[] \texttt{git branch -M main}
		% \explanationItem{
			% This tells Git that you want to track your changes on a branch called "main".
			% If you do not yet know what a branch is don't worry, it is not necessary to know for this task.
			% }
% 	  \commandItem{git remote add origin <url>}
%    \explanationItem{%
%      This tells Git that there is a remote version of the repository in the Gitlab of the Department for Computer Science.
%      You have to replace \texttt{<url>} by the URL of your Gitlab repository.
%      To find that URL, open the Gitlab website and navigate to your repository.
%      Then click on the \textcolor{blue}{blue \texttt{Code}} button on the top right to reveal the SSH and HTTPS urls.
%			The urls should look similar to this\\
%			\parbox{2cm}{SSH:} \texttt{...}\\
%			\parbox{2cm}{HTTPS:} \texttt{...}\\
%      where \texttt{X} is replaced by a certain number.
%      You can use either SSH or HTTPS to connect to your Gitlab remote repository.
%			Both is fine but we recommend SSH because with HTTPS you must enter your credential whenever you want to push.
%			You can find instructions for setting up SSH communication here: \url{https://docs.gitlab.com/ee/user/ssh.html}.
%      }
  	\commandItem{git push -u origin main}
		\explanationItem{%
		  Push your changes to the Gitlab repository.
			There might be a warning "\texttt{The authenticity of host 'git.rz.tu-bs.de' can't be established}".
			When the question "\texttt{Are you sure you want to continue connecting (yes/no/[fingerprint])?}" appears, simply type \texttt{yes} and press "enter".
			If this is successful, you are done. :)
			You should also be able to see your code on the GitLab website now.
			}
  \end{itemize}
}

\deftask{PracticeIntegrateOtherFeature}
{Integrate Feature Developed by Fellow Student}{
	The other students developed some fancy features as well.
	Further improve your calculator by integrating a feature developed by a fellow student.

	You have been granted read access to two repositories of your fellow students (if not yet, you should receive an e-mail soon).
	Similar to your repository, these repositories are called \texttt{Calculator}$n$ and \texttt{Calculator}$m$, where $n, m \in \mathbb{N}, n \neq m$.
	One of these repositories has a smaller number and one of them has a larger number.
	For the rest of this sheet, we hence assume $n < m$.
	Similar to your calculator, these calculators implemented a new feature as well.

	For this task, we want you to integrate the feature developed in \texttt{Calculator}$n$ (with the smaller number) to your project.
	Therefore, you should clone the \texttt{Calculator}$n$ repository, locate the added feature (hint: look at the \texttt{README.md} file of your fellow student) and integrate the respective changes to your calculator, too.
	You may run your peer's calculator first, to have a look at their new feature.
	While working on the integration, make notes on your experience and, in particular, on the pitfalls that came up (cf.\ Exercise 3).
	When you run your calculator afterwards, it should contain both your initial feature as well as the new one from your fellow student, and both features should behave as intended.
	Note that the other branch may have developed the same feature as you did.
	In this case, carefully check if your implementation fully covers the changes made by your colleague.

	Afterward, go back to your \texttt{README.md} and also document the feature you added, similar as you did for your initial feature in the first task.
	If you did not adapt the implementation because you already implemented the feature, also clarify this in the \texttt{README.md}.

	\textbf{Important:} The updated source code and \texttt{README.md} need to be pushed to \emph{your} repository by the submission deadline.
}

\deftask{PracticeRemoveOtherFeature}
{Remove Feature Excluded by Fellow Student}{
	Each student also removed one feature from their calculator.
	Follow the lead of one of your peers and also remove the corresponding feature by applying the changes they made for the removal.
	For this task, use the other repository you have been granted read access to (i.e., \texttt{Calculator}$m$, where $m$ is the larger number).
	Locate the feature removal (hint: look at the \texttt{README.md} file of your fellow student) and integrate the respective changes to your calculator.
	While working on the integration, make notes on your experience and, in particular, on the pitfalls that came up (cf. Exercise 3).
	Be careful of interactions with the previous changes to your calculator.
	Note that the other branch may have removed the same feature as you did.
	In this case, carefully check if your implementation fully covers the changes made in their branch.

	Update the \texttt{README.md} to also indicate the second feature you deleted, analogous to the first task.
	If you did not adapt the implementation because you already removed the same feature, also clarify this in the \texttt{README.md}.

	\textbf{Important:} The updated source code and \texttt{README.md} need to be pushed to \emph{your} repository by the submission deadline.
}

\deftask{PracticeCalculatorShareExperiences}
{Discuss your experience}{
	In the corresponding exercise, we want to discuss the experiences when integrating other features.
	Please prepare answers to the following questions for the discussion:
	\begin{enumerate}
		\item Did you need to edit the same part of the source code for multiple features?
		\item Which problems came up when integrating or removing other features?
		\item Could those problems have been avoided? If so, how?
		\item Do you think that your initially implemented feature was easy to integrate? Why?
	\end{enumerate}
}

\deftask{PracticeDevelopSPLsIntroduction}
{Introduction}{
	The aim of the remaining practical tasks is to gain experience with various implementation techniques for software product lines.
	You are free to choose the domain to be implemented as long as it meets the following criteria:
	\begin{itemize}
		\item The domain contains at least 20 features.
		\item At least 10 of these features can be implemented with reasonable effort (e.g., 5--10 lines of code per feature).
	\end{itemize}

	When selecting the domain, it is permitted to use existing source code as long as it is not yet available as a software product line.
	For example, you can build on Java projects from previous courses or existing open source projects.
	As a programming language you must use Java, as it has the best tool support the upcoming implementation techniques.
	Your domain should describe a \emph{software} product line so that it can be implemented as a software tool or library later.
	(In particular, the domain should not be for a \emph{hardware} product lines such as sandwiches, bikes, or cars which cannot be realized as pure software.)

	The tasks will be implemented with FeatureIDE, a development tool for software product lines.
	To get FeatureIDE, head to its website\footnote{\url{https://featureide.github.io/}} and scroll down to the \emph{Downloads} section and look for the \emph{Prepackaged Versions}.
	There is a table with downloads of prepackaged Eclipse IDEs with certain versions of Eclipse.
	Download \emph{Eclipse with FeatureIDE, JDT, CDT, and AJDT} with the newest version of FeatureIDE (first row) for your operating system.

	A separate FeatureIDE project must be created for each practical task.
	Make sure that you always create a new project and do not simply modify the previous one.
	You should think of a project name that is as unique as possible to avoid conflicts with your fellow students.

	To submit your solution, you will be granted access to a new Gitlab repository soon.\footnote{If you did not get access yet, please wait a day or two. Otherwise, you might have failed the first two practical tasks. If in doubt, send an e-mail to
	\href{mailto:REDACTED?subject=[SPL] Question}{REDACTED}}.
	For checking your submissions and for easy comparison in the exercise \textbf{each project must be named according to the naming pattern \texttt{[project name]-Task[task number]}}!
	For example, if your domain is called \texttt{VideoPlayer}, then the project of the first practical task \textbf{must} be called \texttt{VideoPlayer-Task1}, and after having done all tasks, the top level directory of your repository should look like this:

	\begin{center}
	\includegraphics[width=0.7\linewidth]{expected_repo_root}
	\end{center}

	where again \texttt{VideoPlayer} would be replaced by the name of your domain.
	Each directory \texttt{VideoPlayer-TaskN} contains the FeatureIDE project of the respective task.

	\textbf{Except for Tasks 1 and 2, each submission must have a README file with the following contents:}
	\begin{itemize}
		\item \textbf{Encountered Bugs} in your own project (or FeatureIDE).
			Fixing those bugs is not mandatory.
			%		\item welche der als konkret markierten Features sie in dieser Aufgabe neu implementiert haben und welche Sie aus vorherigen Aufgaben angepasst oder übernommen haben.
		\item Instructions on how you to start and operate your program.
			Name the location of your \texttt{main} method, including \textbf{package name and class name}!
			For example, if the \texttt{main} method is in the \texttt{Start} class in the \texttt{en.upb.spl} package, please reference \texttt{en.uulm.spl.Start} in your README.
		\item If you are building on existing software, please specify its \textbf{source} (e.g., link to website, repository, or archive) and \textbf{license}.
	\end{itemize}
}

\deftask{PracticeDevelopSPLDomainAnalysis}
{Domain Analysis}{
	The first task is to carry out a domain analysis and document the result in a feature diagram using FeatureIDE:
	\begin{itemize}
		\item Create a new FeatureIDE project for this (Composer: \texttt{Feature Modeling}).
		\item In the feature diagram, create at least 20 features and enter a short description for each, using the context menu entry \texttt{Change Description}.
		\item Create at least \textbf{five} typical configurations for your domain.
	\end{itemize}
	\textbf{In later tasks, only letters are allowed in feature names and other characters should be avoided}.

	In the exercise, you will be asked to briefly introduce your domain.
	How did you go about deriving the five typical configurations?
	Why did you choose this domain?
	Is there already an implementation that you would like to use?
}

\newcommand{\markabstract}{%
	Mark the features you implemented in the feature model as concrete and all other features as abstract.
}
\newcommand{\test}{%
	Test your implementation with the five typical configurations and document any \textbf{errors found} in the README file for this task.
	Fixing the errors is optional.
}
\newcommand{\reimplementOldFeatures}{%
	These should also include the features from the previous tasks.
	In any case, avoid mixing different implementation techniques so that all implemented features can be selected via the configurations in FeatureIDE.
}
\newcommand{\experiences}{%
	In this exercise, we are interested in your \textbf{experience}.
	What errors did you find during testing?
}

\deftask{PracticeDevelopSPLRuntimeVariability}
{Runtime Variability}{
	Implement the variability of the previously modeled domain using runtime parameters.
	It should be possible to select and deselect \textbf{at least 2 features} of your domain using \textbf{parameters of the \texttt{main} method}.
	\markabstract{}
	\test{}

	The application should be developed as a \textbf{FeatureIDE project} (Composer: \texttt{Runtime Parameters}) or as a \textbf{Java project} in Eclipse.
	\texttt{HelloWorld-RuntimeProperties} and \texttt{HelloWorld-RuntimeParameters} are example projects that present two possible options for implementing variability with runtime parameters in FeatureIDE.
	Screencasts are also available.\footnote{\url{https://www.youtube.com/watch?v=LjGonbYD7Po} and \url{https://www.youtube.com/watch?v=JxejI0Abtn4}}
	There is currently slightly better tool support for properties (e.g.\ coloring of source code, collaboration diagram).

	Please make sure that your changes have actually been uploaded to the repository, including the \texttt{.project} file (e.g.\ \texttt{push} in Git).

	\experiences{}
	How did you program your application?
	Did you develop the parameters one after the other?
	Which of the five configurations behave identically?
}

\deftask{PracticeDevelopSPLPreprocessing}
{Preprocessors}{
	Implement the variability of the previously modeled domain with the preprocessor.
	It should be possible to select and deselect \textbf{at least 4 features} of your domain.
	\reimplementOldFeatures{}
	Make use of the advantages of preprocessors compared to runtime parameters.
	Do not simply replace every \texttt{if} statement by a preprocessor annotation!
	Can you avoid dead code?

	Develop your application as a \textbf{FeatureIDE project} (Composer: \texttt{Antenna}) in Eclipse.
	Document again in the feature diagram which features are already used in the implementation: Mark all features as \textbf{abstract} that are not implemented or do not require artifacts.
	Mark all features as \textbf{concrete} that actually appear in preprocessor directives.

	\test{}
	How did you localize places where the new features had to be inserted?
	Are there nested annotations?
	What is the maximum number of annotations for a feature?
	Did you find any complicated presence conditions?
}

\deftask{PracticeDevelopSPLBlackboxFramework}
{Black-Box Frameworks}{
	Implement the variability of the previously modeled domain as a framework.
	It should be possible to select and deselect \textbf{at least 6 features} of your domain.
	\reimplementOldFeatures{}
	\markabstract{}
	\test{}

	Develop your application as a \textbf{FeatureIDE project} (Composer: \texttt{Framework}) or in a (plain) \textbf{Java project} in Eclipse.
	If you use a Java project, it should be possible to adapt the configuration by either (1) specifying selected features as command-line arguments (for the \texttt{main} method) or (2) by commenting out \textbf{a single line} per feature, which adds the selected features plugin to a list of selected plugins, in the \texttt{main} method (i.e.\ no plugin loader needs to be implemented).

	If possible, avoid creating a separate interface for each feature.
	However, not all plugins should be integrated via a single interface.
	(If you encounter problems with the plugin loader of the FeatureIDE composer, please use the Q\&A forum or switch to a plain Java project.)

	\experiences{}
	How time-consuming is it to create plugins?
	How often did you have to adapt the framework or the interfaces afterwards?
	How many interfaces are there in your implementation?
	What adjustments might be necessary when adding further plugins?
}

% Practical tasks in use at University of Magdeburg

\deftask{PracticeChart}
{\t{%
	A Chart of Implementation Techniques
}{%
	Implementierungstechniken-Schaubild
}}{
	\t{
		Develop an infographic, diagram, or poster that summarizes and explains the most important relationships regarding the implementation techniques covered in this lecture.
		Which relationships and concepts you consider important and which presentation form you choose is up to you, as long as you can explain your graphic.
		You can limit yourself to the relationships discussed in the lecture and exercise, but you can also incorporate your own thoughts and ideas.
		For example, what conclusion you personally draw from this section of the lecture.
		Below you will find some suggestions that you can consider when designing your graphic (but don't have to).
	}{
		Entwickle eine Infografik bzw. ein Schaubild oder Poster, welche(s) die wichtigsten Zusammenhänge rund um die in dieser Vorlesung behandelten Implementierungstechniken zusammenfasst und erklärt.
		Welche Zusammenhänge und Begriffe du als wichtig erachtest und welche Darstellungsform du wählst, bleibt dir überlassen, so lange du deine Grafik erklären kannst.
		Du kannst dich auf die in der Vorlesung und Übung besprochenen Zusammenhänge beschränken, aber auch deine eigenen Gedanken und Ideen einfließen lassen.
		Zum Beispiel welches Fazit du persönlich aus diesem Abschnitt der Vorlesung ziehst.
		Im Folgenden findest du einige Anregungen, die du bei der Gestaltung deiner Grafik berücksichtigen kannst (aber nicht musst).
	}
	\begin{enumerate}
	\item \t{possible presentation forms:}{mögliche Darstellungsformen:}
		\begin{enumerate}
		\item \t{
			Concept/Mind Map, ontology/semantic network, organizational chart, flowchart, history/timeline, pictograms/illustrations, mosaic, \ldots
		}{
			Concept/Mind Map, Ontologie/semantisches Netz, Organigramm, Flussdiagramm, Historie/Zeitstrahl, Piktogramme/Illustrationen, Mosaik, \ldots
		}
		\end{enumerate}
	\item \t{possible relationships:}{mögliche Zusammenhänge:}
		\begin{enumerate}
		\item \t{
			\emph{Dimensions of variability} such as binding time, tool-/language-based, annotative/compositional, static/dynamic
		}{
			\emph{Dimensionen der Variabilität} wie Bindungszeitpunkt, tool-/sprachbasiert, annotativ/kompositional, statisch/dynamisch
		}
		\item \t{
			\emph{Quality criteria} such as feature traceability, separation of concerns, information hiding/cohesion/encapsulation, granularity, uniformity, obliviousness
		}{
			\emph{Qualitätskriterien} wie Feature Traceability, Separation of Concerns, Information Hiding/Kohäsion/Kapselung, Granularität, Uniformität, Obliviousness
		}
		\item \t{
			\emph{Problems} such as code tangling/scattering, code replication/cloning, preplanning, crosscutting concerns, fragile evolution
		}{
			\emph{Probleme} wie Code Tangling/Scattering, Code Replication/Cloning, Preplanning, querschneidende Belange, Fragile Evolution
		}
		\item \t{
			\emph{Scenarios} such as safety- or performance-critical systems, fine-grained extensions, adoption strategy, embedded systems, training effort, scalability
		}{
			\emph{Szenarien} wie sicherheits- oder performance-kritische Systeme, feingranulare Erweiterungen, Adoptionsstrategie, eingebettete Systeme, Schulungsaufwand, Skalierbarkeit
		}
		\item \t{and possibly more}{und ggf. weitere}
		\end{enumerate}
	\end{enumerate}
	\t{
		When in doubt, follow the Unix philosophy \emph{do one thing and do it well}.
		That is, don't try to explain \emph{all} relationships (which would probably lead to an overloaded and complex graphic), but rather some selected ones (which you then explain meaningfully and in sufficient detail and put into context).
		Tables are not admitted as a solution.
	}{
		Verfolge im Zweifelsfall die Unix-Philosophie \emph{do one thing and do it well}.
		Das heißt, versuche nicht, \emph{alle} Zusammenhänge zu erläutern (was vermutlich zu einer überladenen und komplexen Grafik führt), sondern einige ausgewählte (die du dann sinnvoll und hinreichend detailliert erklärst und in Bezug setzt).
		Tabellen sind als Lösung nicht zugelassen.
	}
	\begin{solution}
REDACTED
	\end{solution}
}

\deftask{PracticeDevelopSPL}
{Implementierung einer Software-Produktlinie}{
	\emph{Für Bachelorstudierende ist diese Aufgabe fakultativ (Bonus von +5 Votierungspunkten).}

	Das Ziel dieser Aufgabe ist es, Erfahrungen mit einer (oder mehreren) Implementierungstechnik für Software-Produktlinien zu sammeln.
	Deine Lösung sollte den folgenden Kriterien entsprechen.
	Bei Fragen gilt: Bitte melde dich so früh wie möglich.
	\begin{itemize}
		\item \textbf{Implementierungstechnik}~~
		Du kannst frei aus folgenden Implementierungstechniken wählen:
		Laufzeitvariabilität (z.B. Parameter oder Design Patterns), Prä\-prozessoren, Buildsysteme, Komponenten, Services, Frameworks, Feature-orien\-tier\-te Programmierung und aspektorientierte Programmierung.
		Wo du es für sinnvoll und technisch umsetzbar hältst, sind auch Kombinationen erlaubt.
		Mit Absprache sind auch andere Ansätze erlaubt.
		Clone-and-Own ist bei dieser Aufgabe unerwünscht -- das macht ihr nach dem Studium noch genug \faSmileWink[regular]
		\item \textbf{Werkzeuge}~
		Dir steht frei, welche konkreten Werkzeuge du zur Umsetzung wählst.
		Auch die Programmiersprache kannst du frei wählen.
		Wichtig ist, dass der Code am Ende ausführbar ist und ``auf Knopfdruck'' verschiedene Produkte generiert werden können.
		Gängige Werkzeugkombinationen könnten zum Beispiel sein: FeatureIDE + Java + \ldots, C + KConfig + KBuild oder AspectJ + Java.
		\item \textbf{Domäne}~~
		Du kannst die Domäne frei wählen, es soll sich aber explizit \emph{nicht} um ein Beispiel aus der Vorlesung oder einer der Übungsaufgaben handeln (also z.B. keine Graphbibliotheken, Taschenrechner oder FeatureIDE-Beispiele).
		Hier ein paar Anregungen in Form von Domänen, die in früheren Jahren von Studierenden verwendet wurden:
		Shopping, Spiele (z.B. Text-Adventures, Tic Tac Toe, \ldots), Wetter, das römische Reich, Kalender, Datum/Uhrzeit, Arduino-Sensordaten, \ldots
		\item \textbf{Features}~~
		Deine Software-Produktlinie sollte $\ge 10$ Features enthalten, die jeweils sowohl an- als auch abwählbar (also frei von Anomalien) sind.
		Nicht alle diese Features sollten vollständig optional und unabhängig sein, d.h. es sollte auch Features geben, die von anderen Features abhängen oder anderweitige Constraints erfordern.
		Diese Constraints sollten im Code berücksichtigt werden (z.B. durch ein Feature-Modell, falls du ein solches verwendest), so dass keine ungültigen Produkte generiert werden können.
		Es sollte aber auch nicht so viele Constraints geben, dass nur sehr wenige Produkte generiert werden können.
		\item \textbf{Code}~~
		Deine Lösung sollte $\ge 1000$ selbst geschriebene \emph{source lines of code} (SLOC) enthalten.
		Ob der Code explizit für diese Lehrveranstaltung geschrieben wurde oder ob du auf bestehenden Code zurückgreifst, ist unerheblich.
		Wichtig ist, dass es \emph{dein} Code ist (u.a. kein KI-generierter Code, kein fremder Code aus dem Internet oder von Kommiliton*innen -- unter Vorbehalt einer Nachprüfung).
		Am einfachsten ist es, ein komplett neues Projekt zu beginnen.
		Es bietet sich aber auch an, Code aus früheren Lehrveranstaltungen zu ``Produktlinien-fizieren'', also nach den in der Lehrveranstaltung dargestellten Prinzipien konfigurierbar zu machen.
		\item \textbf{Abgabe und Vorstellung}~~
		Der Code kann als Anhang per Mail oder als Weblink abgeschickt werden.
		In jedem Fall ist Git zur Versionsverwaltung zu verwenden und der \texttt{.git}-Ordner mit abzugeben.
		Bitte vermeide, nur einen Commit zu erzeugen, damit die Autorenschaft besser nachvollziehbar ist.
		Bereite dich außerdem darauf vor, deine Lösung kurz in der Übung vorzustellen (d.h., den Code ausführen, Produkte konfigurieren, und deine Vorgehensweise erläutern).
		Sowohl Abgabe als auch Vorstellung sind zur erfolgreichen Bearbeitung der Aufgabe notwendig.
	\end{itemize}
}