\deftask{IntroductionBackpackCustomization}
{\t{Tailor-Made vs. Mass-Produced}{Maß oder Masse?}}{
	\t{
		Imagine you are starting a company that designs customizable backpacks.
		You would want that customers can choose different sizes, colors, materials, and extra pockets.
	}{
		Stell dir vor, du gründest ein Unternehmen, das anpassbare Rucksäcke entwirft.
		Kunden sollen verschiedene Größen, Farben, Materialien und zusätzliche Taschen wählen können.
	}
	\begin{enumerate}
		\item \t{
			What is the difference between a tailor-made backpack and a mass-produced backpack?
		}{
			Was ist der Unterschied zwischen einem maßgeschneiderten Rucksack und einem massenproduzierten Rucksack?
		}
		\item \t{
			Would it be smart to offer mass customization for backpacks instead of using only tailor-made or mass production?
			Why or why not?
		}{
			Wäre es sinnvoll, \emph{Mass Customization} (kundenindividuelle Massenproduktion) für Rucksäcke anzubieten, anstelle von Maßanfertigung oder Massenproduktion?
			Warum oder warum nicht?
		}
	\end{enumerate}
	
	\begin{solution}
REDACTED
	\end{solution}
}

\deftask{IntroductionFastfoodCustomization}
{\t{%
	Understanding Features, Configurations, and Dependencies
}{%
	Features, Konfigurationen und Abhängigkeiten verstehen
}}{
	\t{
		You are the proud owner of \textit{Bite \& Build}, a fast-food restaurant famous for its fully customizable meals.
		Every day, your hungry customers walk in, excited to build their own unique meals by mixing and matching the ingredients and options your restaurant offers.
		There are so many possible combinations that even your regular customers can try something new every time!

		Your restaurant's success depends on understanding how these customized orders come together, how the options interact, and how to keep everything organized in the kitchen.
	}{
		Du bist stolze Besitzerin von \textit{Bite \& Build}, einem Fast-Food-Restaurant, das für seine vollständig anpassbaren Mahlzeiten bekannt ist.
		Jeden Tag kommen deine hungrigen Kunden herein, begeistert davon, ihre eigenen einzigartigen Mahlzeiten zu erstellen, indem sie die Zutaten und Optionen mischen und kombinieren, die dein Restaurant anbietet.
		Es gibt so viele mögliche Kombinationen, dass selbst deine Stammkunden jedes Mal etwas Neues ausprobieren können!

		Der Erfolg deines Restaurants hängt davon ab, zu verstehen, wie diese angepassten Bestellungen zusammengestellt werden, wie die Optionen interagieren und wie alles in der Küche organisiert bleibt.
	}
	
	\begin{enumerate}
		\item \t{
			Describe the complete journey from the moment a customer starts building their meal to the moment it is served.
			Use the following terms correctly in your explanation: feature, domain, configuration, and product.
		}{
			Beschreibe den vollständigen Ablauf vom Moment, in dem ein Kunde beginnt, seine Mahlzeit zusammenzustellen, bis zu dem Moment, in dem sie serviert wird.
			Verwende die folgenden Begriffe korrekt in deiner Erklärung: Feature, Domäne, Konfiguration und Produkt.
		}
		\item \t{
			Create a menu by listing at least ten features that customers can select to customize their meals.
		}{
			Erstelle ein Menü, indem du mindestens zehn Features auflistest, die Kunden auswäh\-len können, um ihre Mahlzeiten anzupassen.
		}
		\item \t{
			Some ingredients might depend on each other or might not work together at all.
			Can you think of two or more features that either depend on each other or create a conflict (for example, combinations that cannot be served together)?
		}{
			Einige Zutaten könnten voneinander abhängen oder überhaupt nicht zusammen funktionieren.
			Kannst du dir zwei oder mehr Features vorstellen, die entweder voneinander abhängen oder einen Konflikt erzeugen (zum Beispiel Kombinationen, die nicht zusammen serviert werden können)?
		}
		\item \t{
			Are there differences between selling fast food (or other ``hardware'') and selling software?
		}{
			Gibt es Unterschiede zwischen dem Verkauf von Fast Food (und anderer ``Hardware'') und Software?
		}
	\end{enumerate}
	
	\begin{solution}
REDACTED
	\end{solution}
}

\deftask{IntroductionSoftwareProductLines}
{\t{Software Product Lines}{Software-Produktlinien}}{
	\begin{enumerate}
		\item \t{
			What is a software product line?
		}{
			Was versteht man unter einer Software-Produktlinie?
		}
		\begin{solution}
REDACTED
		\end{solution}
		\item \t{
			In which situations and why should software product lines be used?
		}{
			Wann und warum sollten Software-Produktlinien eingesetzt werden?
		}
		\begin{solution}
REDACTED
		\end{solution}
		\item \t{
			Are the following systems software product lines?
			Justify your answer.
		}{
			Sind die folgenden Systeme Software-Produktlinien?
			Begründe deine Antwort.
		}

		\emph{Linux, Visual Studio Code, HP Printer Driver, Microsoft Office, Spotify, Minecraft}
		
		\begin{solution}
REDACTED
		\end{solution}
		\item \t{
			Give further examples of software systems which are (not) product lines.
		}{
			Nenne weitere Beispiele für Softwaresysteme, die (keine) Produktlinien sind.
		}
	\end{enumerate}
}

\deftask{IntroductionSPLPromises}
{\t{Promises of Software Product Lines}{Vorteile von Software-Produktlinien}}{
	\t{
		Software product lines promise multiple advantages over traditional single-system development.
		Summarize and explain the key benefits of using software product lines in practice.
		Also, describe how up-front investment in product-line engineering pays off over time.
		Support your answer with a concrete example(s).
	}{
		Software-Produktlinien versprechen mehrere Vorteile gegenüber der traditionellen Einzel\-system-Entwicklung.
		Fasse die wichtigsten Vorteile der Verwendung von Software-Produkt\-linien in der Praxis zusammen und erkläre sie.
		Beschreibe auch, wie sich eine anfängliche Investition in die Produktlinienentwicklung im Laufe der Zeit auszahlt.
		Unterstütze deine Antwort mit einem konkreten Beispiel.
	}
	
	\begin{solution}
REDACTED
	\end{solution}
}

\deftask{IntroductionSPLChallenges}
{\t{Challenges of Software Product Lines}{Herausforderungen von Software-Produktlinien}}{
	\t{
		When working with software product lines (SPLs), various situations can occur that require careful attention during development and management.

		Below are six scenarios that can happen when working with software product lines.
		Each scenario is related to one typical challenge that developers commonly face.

		For each of the following scenarios, \textbf{identify} the software product line challenge that the scenario represents and \textbf{briefly explain why this situation might lead to a problem} in the context of software product lines.
	}{
		Bei der Arbeit mit Software-Produktlinien (SPLs) können verschiedene Situationen auftreten, die während der Entwicklung und Verwaltung große Sorgfalt erfordern.

		Im Folgenden findest du sechs Szenarien, die bei der Arbeit mit Software-Produktlinien auftreten können.
		Jedes Szenario steht in Verbindung mit einer typischen Herausforderung, der Entwickler häufig begegnen.

		\textbf{Identifiziere} für jedes der folgenden Szenarien die Software-Produktlinien-Herausforde\-rung, die das Szenario darstellt, und \textbf{erkläre kurz, warum diese Situation im Kontext von Software-Produktlinien zu einem Problem führen könnte}.
	}
	
	\begin{enumerate}
		\item \t{
			A company rapidly copies the entire source code of a successful product to meet the urgent request of a different customer in another country.
			To save time, they create a second product with minor changes in configuration files.
		}{
			Ein Unternehmen kopiert schnell den gesamten Quellcode eines erfolgreichen Produkts, um der dringenden Anfrage eines anderen Kunden in einem anderen Land nachzukommen.
			Um Zeit zu sparen, erstellen sie ein zweites Produkt mit geringfügi\-gen Änderungen in den Konfigurationsdateien.
		}

		\item \t{
			In a software project that supports regional market variations, dozens of optional features are activated depending on local regulations.
			After several development cycles, it becomes difficult for teams to understand how individual features are implemented in the source code.
		}{
			In einem Softwareprojekt, das regionale Marktvariationen unterstützt, werden je nach lokalen Vorschriften Dutzende optionaler Features aktiviert.
			Nach mehreren Entwicklungszyklen wird es für Teams schwierig zu verstehen, wie einzelne Features im Quellcode implementiert sind.
		}

		\item \t{
			A company introduces an advanced build system that automatically generates product variants based on feature selections.
			However, developers often struggle to successfully assemble complete products from selected features.
		}{
			Ein Unternehmen führt ein fortgeschrittenes Build-System ein, das automatisch Produktvarianten basierend auf Feature-Auswahlen generiert.
			Entwickler haben jedoch häufig Schwierigkeiten, vollständige Produkte aus ausgewählten Features erfolgreich zusammenzustellen.
		}

		\item \t{
			A smartphone manufacturer offers customers the option to configure their phones online by selecting up to 20 optional features like waterproofing, extra storage, face recognition, multiple camera setups, or gaming optimizations.
			The company provides hundreds of different customer-specific phones each year.
		}{
			Ein Smartphone-Hersteller bietet Kunden die Möglichkeit, ihre Telefone online zu konfigurieren, indem sie bis zu 20 optionale Features wie Wasserdichtigkeit, zusätzli\-chen Speicher, Gesichtserkennung, mehrere Kamera-Setups oder Gaming-Optimie\-run\-gen auswählen.
			Das Unternehmen stellt jedes Jahr Hunderte verschiedener kundenspezifischer Telefone bereit.
		}

		\item \t{
			A customer configures a custom smart home system by combining voice-controlled lighting with advanced security features.
			Both features are available for all customers and can be selected in any combination using an online configurator.
		}{
			Ein Kunde konfiguriert ein individuelles Smart-Home-System, indem er sprachgesteuerte Beleuchtung mit erweiterten Sicherheitsfunktionen kombiniert.
			Beide Features sind für alle Kunden verfügbar und können in jeder Kombination über einen Online-Konfigurator ausgewählt werden.
		}

		\item \t{
			An electric car company has been evolving its software product line for over ten years, continuously adding new driving modes, safety features, entertainment apps, and country-specific adjustments.
		}{
			Ein Elektroauto-Hersteller entwickelt seit über zehn Jahren seine Software-Produktli\-nie weiter und fügt kontinuierlich neue Fahrmodi, Sicherheitsfunktionen, Unterhal\-tungs-Apps und länderspezifische Anpassungen hinzu.
		}
	\end{enumerate}
	
	\begin{solution}
REDACTED
	\end{solution}
}