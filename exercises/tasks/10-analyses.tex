\deftask{AnalysesStrategies}
{\t{Analysis Strategies for Software Product Lines}{Analyse-Strategien für Software-Produktlinien}}{
	\t{
		Define and explain the three primary analysis strategies used in software-product-line engineering.
		For each strategy, discuss its main characteristics, benefits, and drawbacks using concrete examples.
		Which strategy would you recommend under which circumstances?
	}{
		Definiere und erkläre die drei primären Analyse-Strategien, die bei der Entwicklung von Software-Produktlinien verwendet werden.
		Diskutiere für jede Strategie ihre Hauptmerkmale, Vorteile und Nachteile anhand konkreter Beispiele.
		Welche Strategie würdest du unter welchen Umständen empfehlen?
	}
	\begin{solution}
REDACTED
	\end{solution}
}

\deftask{AnalysesProblemSolutionSpace}
{\t{%
	Analyzing Disagreements Between Problem and Solution Space
}{%
	Analyse von Diskrepanzen zwischen Problem- und Lösungsraum
}}{
	\t{
		Consider the following product line for secure communication:
	}{
		Gegeben sei folgende Produktlinie für sichere Kommunikation:
	}
	\myexample{\t{%
		Feature Model
	}{%
		Feature-Modell
	}}{
		\centering\sffamily
		\featureDiagram{
			SecureComm,abstract
			[TLS,optional,concrete]
			[AUTH,optional,concrete]
			[LOGGING,optional,concrete]
		}
		\\
		$\text{TLS} \pimplies \text{AUTH}$ \\
		$\text{LOGGING} \pimplies \text{TLS}$
	}
	\myexample{\t{%
		Annotated Source Code
	}{%
		Annotierter Quelltext
	}}{
		\includegraphics[width=0.45\linewidth]{code-prob-sol-securecomm}
	}

	\begin{enumerate}
		\item \t{
			What is the problem space of a software product line (in general and in this example)?
			What is the solution space?
			What problems can arise when developers only consider one of them when conducting an analysis?
		}{
			Was ist der Problemraum einer Software-Produktlinie (allgemein und in diesem Beispiel)?
			Was ist der Lösungsraum?
			Welche Probleme können auftreten, wenn Entwickler bei der Analyse nur einen von beiden betrachten?
		}
		\begin{solution}
REDACTED
		\end{solution}

		\item \t{
			Which configurations are valid in the given problem space?
		}{
			Welche Konfigurationen sind im gegebenen Problemraum gültig?
		}
		\begin{solution}
REDACTED
		\end{solution}

		\item \t{
			Which configurations compile successfully in the solution space?
		}{
			Welche Konfigurationen kompilieren erfolgreich im Lösungsraum?
		}
		\begin{solution}
REDACTED
		\end{solution}

		\item \t{
			Give one configuration that is valid in the problem space but fails to compile in the solution space.
		}{
			Gib eine Konfiguration an, die im Problemraum gültig ist, aber im Lösungsraum nicht kompiliert.
		}
		\begin{solution}
REDACTED
		\end{solution}

		\item \t{
			Give one configuration that is invalid in the problem space but compiles in the solution space.
		}{
			Gib eine Konfiguration an, die im Problemraum ungültig ist, aber im Lösungsraum kompiliert.
		}
		\begin{solution}
REDACTED
		\end{solution}

		\item \t{
			Propose a sensible fix for the discrepancy between problem and solution space in this example.
		}{
			Schlage vor, wie die Diskrepanz zwischen Problem- und Lösungsraum in diesem Beispiel sinnvoll behoben werden kann.
		}
		\begin{solution}
REDACTED
		\end{solution}
	\end{enumerate}
}

\deftask{AnalysesFeatureMapping}
{\t{%
	Analyzing Feature Mappings in Preprocessor-Based Product Lines
}{%
	Analyse von Feature Mappings in Präprozessor-basierten Produktlinien
}}{
	\begin{enumerate}
		\item \t{
			What is dead code in a preprocessor-based product line?
			What are superfluous annotations?
			Explain why it is beneficial to know about the presence of these anomalies in a code base.
		}{
			Was ist toter Code in einer Präprozessor-basierten Produktlinie?
			Was sind über\-flüs\-si\-ge Annotationen?
			Erkläre, warum es vorteilhaft ist, über das Vorhandensein dieser Anomalien in einer Codebasis Bescheid zu wissen.
		}
		\begin{solution}
REDACTED
		\end{solution}

		\item \t{
			Which of both problems occur in the following source code, where, and why?
			Justify your answer using presence conditions.
		}{
			Welche dieser beiden Probleme treten im folgenden Quellcode auf, wo, und warum?
			Begründe deine Antwort anhand von \emph{Presence Conditions}.
		}
		\begin{center}
			\myexample{\t{%
				Conditionally Compiled \texttt{Graph} Code
			}{%
				Bedingt kompilierter \texttt{Graph}-Code
			}}{
				\includegraphics[width=0.65\linewidth]{dead-superfluous-code}
			}

		\end{center}
		\begin{solution}
REDACTED
		\end{solution}

		\item \t{
			Given any (unpreprocessed) code fragment, how can we automatically detect dead code and superfluous annotations?
		}{
			Angenommen, ein beliebiger Codeausschnitt (vor Ausführung des Präprozessors) ist gegeben.
			Wie kann man automatisiert toten Code und überflüssige Annotationen erkennen?
		}
		\begin{solution}
REDACTED
		\end{solution}

		\item \t{
			Should the problem space be incorporated into the automated detection procedure?
			If so, how?
			If not, why not?
		}{
			Sollte der Problemraum in das automatisierte Erkennungsverfahren einbezogen werden?
			Wenn ja, wie?
			Wenn nein, warum nicht?
		}
		\begin{solution}
REDACTED
		\end{solution}
	\end{enumerate}
}

\deftask{AnalysesVariableCode}
{\t{%
	Analyzing Variable Code in Preprocessor-Based Product Lines
}{%
	Analyse von variablem Code in Präprozessor-basierten Produktlinien
}}{
	\t{
		Given below is the feature model and code for variable data structures:
	}{
		Gegeben sei folgendes Feature-Modell mit Code für variable Datenstrukturen:
	}

	\myexample{\t{%
		Feature Model
	}{%
		Feature-Modell
	}}{
		\centering\sffamily
		\featureDiagram{
			DataStructures,abstract
			[Algorithms,optional,concrete[QuickSort,concrete,or][LinearSearch,concrete]]
			[Structures,optional,concrete[Array,optional,concrete][Tree,optional,concrete]]
			[Visualization,mandatory,concrete[Simulation,mandatory,concrete]]
		}
		\\
		$\text{QuickSort} \lor \text{LinearSearch} \rightarrow \text{Array}$
	}

	\myexample{\t{%
		Annotated Source Code
	}{%
		Annotierter Quelltext
	}}{
		\includegraphics[width=0.55\linewidth]{sorting-algorithm}
	}

	\begin{enumerate}
		\item \t{
			Considering both the problem and solution space, are there dead code fragments or superfluous annotations in this example?
		}{
			Gibt es toten Quelltext oder überflüssige Annotationen in diesem Beispiel?
			Beziehe dabei sowohl den Problem- als auch den Lösungsraum mit ein.
		}
		\begin{solution}
REDACTED
		\end{solution}
		
		\item \t{
			In general, how can unreachable references and conflicting definitions be detected with a SAT solver?
			Check whether such anomalies appear in this example.
		}{
			Wie kann man erreichbare Referenzen und Definitionskonflikte mit einem SAT-Solver allgemein untersuchen?
			Überprüfe, ob solche Anomalien in diesem Beispiel auftreten.
		}
		\begin{solution}
REDACTED
		\end{solution}
		
		\item \t{
			Can you detect any additional errors in the given example that prevent compiling?
			Which configurations provoke the found errors?
		}{
			Findest du weitere Fehler in dem gegebenen Beispiel, die das Kompilieren verhindern?
			In welchen Konfigurationen treten die gefundenen Fehler auf?
		}
		\begin{solution}
REDACTED
		\end{solution}
	\end{enumerate}
}

\deftask{AnalysisComplexity}
{\t{%
	NP-Completeness of SAT: Theory and Practice
}{%
	NP-Vollständigkeit von SAT: Theorie und Praxis
}}{
	\t{
		To answer this task, you may read Sections 1, 4, 5, and 10 of the paper attached below:
	}{
		Lies zur Bearbeitung ggf. die Abschnitte 1, 4, 5 und 10 des unten angehängten Papers:
	}

	\emph{Marcilio Mendonca, Andrzej Wasowski, and Krzysztof Czarnecki. 2009.
	SAT-based Analysis of Feature Models Is Easy.
	In SPLC.
	Carnegie Mellon University, USA, 231--240.}
	
	\begin{enumerate}
	\item \t{
		The paper discusses a contradictory phenomenon with respect to the satisfiability problem of propositional logic (SAT) and feature models.
		Where lies this contradiction, and how do the authors explain it?
	}{
		In dem Paper wird ein auf den ersten Blick widersprüchliches Phänomen in Bezug auf das Erfüllbarkeitsproblem der Aussagenlogik (SAT) und Feature-Modelle untersucht.
		Worin liegt der Widerspruch und wie erklären ihn sich die Autoren?
	}

	\item \t{
		According to the authors, how can the `difficulty' of a SAT instance (i.e., a propositional formula) be characterized?
		Which SAT instances are particularly difficult?
		Also refer to the terms \emph{transition phase} and \emph{crossover point}.
	}{
		Wie kann man die ``Schwierigkeit'' einer SAT-Instanz (also einer aussagenlogischen Formel) laut den Autoren charakterisieren?
		Welche SAT-Instanzen sind besonders schwierig?
		Nimm auch Bezug auf die Begriffe \emph{transition phase} und \emph{crossover point}.
	}

	\item \t{
		Why is it still difficult to determine the \emph{number} of all valid feature configurations by using a SAT solver?
	}{
		Warum ist es dennoch schwierig, mit einem SAT-Solver die \emph{Anzahl} valider Konfigurationen in einem Feature-Modell zu zählen?
	}
	\end{enumerate}
}