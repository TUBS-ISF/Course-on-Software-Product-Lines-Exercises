\deftask{RuntimeVariabilityConfiguration}
{\t{Configuring Runtime Variability}{Laufzeitvariabilität konfigurieren}}{
	\begin{enumerate}
		\item \t{What is runtime variability?}{Was ist Laufzeitvariabilität?}
		\begin{solution}
REDACTED
		\end{solution}
		
		\item \t{
			Provide examples of two real-world software systems that make use of runtime variability.
			Describe how they use it and how the user configures them.
		}{
			Nenne Beispiele für zwei reale Softwaresysteme, die Laufzeitvariabilität nutzen.
			Beschreibe, wie sie diese nutzen und wie der Benutzer sie konfiguriert.
		}
		\begin{solution}
REDACTED
		\end{solution}
		
		\item \t{
			Imagine you are developing a music player app that offers the following configurable options: visual equalizer (enabled or disabled), streaming quality (high or low), and theme (light or dark).
		}{
			Stell dir vor, du entwickelst eine Musik-Player-App mit folgenden konfigurierbaren Optionen: visueller Equalizer (aktiviert oder deaktiviert), Streaming-Qualität (hoch oder niedrig) und Theme (hell oder dunkel).
		}
		\begin{enumerate}
			\item \t{
				Describe two possible methods to configure these options at runtime.
			}{
				Beschreibe zwei mögliche Methoden, um diese Optionen zur Laufzeit zu konfigurieren.
			}
			\begin{solution}
REDACTED
			\end{solution}
			
			\item \t{
				Using a suitable example, explain what could happen if conflicting options are selected.
				Describe how you would detect and handle such conflicts at runtime.
			}{
				Erkläre an einem geeigneten Beispiel, was passieren könnte, wenn widersprüch\-liche Optionen ausgewählt werden.
				Beschreibe, wie du solche Konflikte zur Laufzeit erkennen und behandeln würdest.
			}
			\begin{solution}
REDACTED
			\end{solution}
		\end{enumerate}
			
		\item \t{
			What are the pros and cons of realizing variants with runtime variability?
		}{
			Welche Vor- und Nachteile hat die Umsetzung von Varianten mit Laufzeitvariabilität?
		}
		\begin{solution}
REDACTED
		\end{solution}
	\end{enumerate}
}

\deftask{RuntimeVariabilityRealization}
{\t{Realizing Runtime Variability}{Laufzeitvariabilität realisieren}}{
	\t{
		You are developing a modern navigation app used by thousands of daily commuters, tourists, and fitness enthusiasts.
		The app must support three primary modes of operation: driving mode, walking mode, and cycling mode.
		Each mode provides a different user experience, route calculation, and interface adjustment tailored to the user's current activity.
		Users may switch between these modes depending on their environment and preferences to receive the most accurate and relevant navigation instructions.
	}{
		Du entwickelst eine moderne Navigations-App, die von Tausenden von Pendlern, Touristen und Fitness-Enthusiasten täglich genutzt wird.
		Die App muss drei primäre Betriebsmodi unterstützen: Fahrmodus, Gehmodus und Fahrradmodus.
		Jeder Modus bietet eine unterschiedliche Benutzererfahrung, Routenberechnung und Schnittstellenanpassung, die auf die aktuelle Aktivität des Benutzers zugeschnitten ist.
		Benutzer können je nach Umgebung und Präferenzen zwischen diesen Modi wechseln, um die genauesten und relevantesten Navigationsanweisungen zu erhalten.
	}
	\begin{enumerate}
		\item \t{
			Describe how you would implement this variability using global parameters.
		}{
			Beschreibe, wie du diese Variabilität mit globalen Parametern umsetzen würdest.
		}
		\begin{solution}
REDACTED
		\end{solution}
		
		\item \t{
			Describe an alternative solution using method parameters instead of global parameters.
		}{
			Beschreibe eine alternative Lösung mit Methodenparametern anstelle von globalen Parametern.
		}
		\begin{solution}
REDACTED
		\end{solution}
		
		\item \t{
			Compare the two solutions.
			Which one is easier to maintain and which one is more flexible for future updates?
			Justify your answer.
		}{
			Vergleiche die beiden Lösungen.
			Welche ist einfacher zu warten und welche ist flexibler für zukünftige Updates?
			Begründe deine Antwort.
		}
		\begin{solution}
REDACTED
		\end{solution}
		\newpage
		\item \t{
			Based on your preferred approach, implement the navigation logic using the scaffold below.
			Explain which variability approach you used (global configuration, method parameters, or design pattern) and why it is suitable in this case.
		}{
			Implementiere basierend auf deinem bevorzugten Ansatz die Navigationslogik unter Verwendung des unten stehenden Gerüsts.
			Erkläre, welchen Variabilitätsansatz du verwendet hast (globale Konfiguration, Methodenparameter oder Design Patterns) und warum er in diesem Fall geeignet ist.
		}
		
		\myexample{
			\t{Scaffold for implementing navigation variability}{Gerüst zur Implementierung von Navigationsvariabilität}
		}{
			\includegraphics[scale=1.2]{code-navigation-stub}
		}
		
		\begin{solution}
REDACTED
		\end{solution}
		
	\end{enumerate}
}

\deftask{RuntimeDesignPatterns}
{\t{Design Patterns for Runtime Variability}{Design Patterns für Laufzeitvariabilität}}{
	\t{
		You are designing customizable software for a coffee machine.
		The machine can prepare different drinks such as espresso, cappuccino, and latte.
		The machine can optionally add sugar or milk and can log preparation times to a file if enabled.
	}{
		Du entwirfst anpassbare Software für eine Kaffeemaschine.
		Die Maschine kann verschiedene Getränke wie Espresso, Cappuccino und Latte zubereiten.
		Die Maschine kann optional Zucker oder Milch hinzufügen und Zubereitungszeiten in einer Datei protokollieren.
	}
	\begin{enumerate}
		\item \t{
			Explain \emph{design patterns} and their general purpose!
			Name and briefly explain one design pattern that can be used to implement variability.
		}{
			Erkläre \emph{Design Patterns} und deren allgemeinen Zweck!
			Nenne und erkläre kurz ein Design Pattern, das zur Implementierung von Variabilität verwendet werden kann.
		}
		\begin{solution}
REDACTED
		\end{solution}
	
		\item \t{
			Select one design pattern that you would use to implement the variability in the coffee machine system.
			Justify why this design pattern is appropriate for this scenario.
		}{
			Wähle ein Design Pattern aus, das du verwenden würdest, um die Variabilität im Kaffeemaschinen-System zu implementieren.
			Begründe, warum dieses Design Pattern für dieses Szenario geeignet ist.
		}
		\begin{solution}
REDACTED
		\end{solution}
		
		\item \t{
			Write a small pseudo-code snippet or construct a class diagram demonstrating how the selected design pattern would work in the coffee machine system.
		}{
			Schreibe einen kleinen Pseudo-Code-Ausschnitt oder erstelle ein Klassendiagramm, das zeigt, wie das ausgewählte Design Pattern im Kaffeemaschinen-System funktionieren würde.
		}
		\begin{solution}
REDACTED
		\end{solution}
		
		\item \t{
			Imagine the user changes the coffee recipe while the machine is already preparing the drink.
		}{
			Stell dir vor, der Benutzer ändert das Kaffee-Rezept, während die Maschine das Getränk bereits zubereitet.
		}
		\begin{enumerate}
			\item \t{
				Explain what could go wrong if configurations are changed at runtime during preparation.
			}{
				Erkläre, was schief gehen könnte, wenn Konfigurationen während der Zubereitung zur Laufzeit geändert werden.
			}
			\begin{solution}
REDACTED
			\end{solution}
			
			\item \t{
				Propose and explain a solution to prevent potential failures.
			}{
				Schlage eine Lösung vor, um potenzielle Fehler zu verhindern, und erkläre diese.
			}
			\begin{solution}
REDACTED
			\end{solution}
			
		\end{enumerate}
		\item \t{
			Could you implement this coffee machine system without using a design pattern?
			If yes, how?
			If not, why not?
		}{
			Könntest du dieses Kaffeemaschinen-System ohne Verwendung eines Design Patterns implementieren?
			Wenn ja, wie?
			Wenn nicht, warum nicht?
		}
		\begin{solution}
REDACTED
		\end{solution}
	\end{enumerate}
}

\deftask{RuntimeDesignPatternsGameAudio}
{\t{Parameters Vs. Design Patterns}{Parameter oder Design Patterns?}}{
	\t{
		The following Java source code fragment implements variability at runtime by using global configuration parameters to manage different audio outputs in a video game.
	}{
		Das folgende Java-Quelltextfragment implementiert Variabilität zur Laufzeit mit globalen Konfigurationsparametern, um verschiedene Audio-Ausgaben in einem Videospiel zu ermöglichen.
	}
	
	\myexample{
		\t{A method with configurable audio output in Java}{Eine Methode mit konfigurierbarer Audio-Ausgabe in Java}
	}{
		\includegraphics[scale=1.2]{code-audio-output}
	}
	
	\t{
		The current implementation must be improved to satisfy the following requirements:
	}{
		Die aktuelle Implementierung muss verbessert werden, um die folgenden Anforderungen zu erfüllen:
	}
	\begin{itemize}
		\item \t{
			The system should allow dynamically switching audio outputs during gameplay without restarting the game.
		}{
			Das System sollte das dynamische Umschalten von Audio-Ausgaben während des Spielens ermöglichen, ohne das Spiel neu zu starten.
		}
		\item \t{
			The system should support playing sounds on multiple outputs at the same time.
		}{
			Das System sollte das gleichzeitige Abspielen von Tönen auf mehreren Ausgaben unterstützen.
		}
		\item \t{
			The system should allow adding new audio output types, such as VR headsets, without modifying the existing source code.
		}{
			Das System sollte das Hinzufügen neuer Audio-Ausgabe-Typen wie VR-Headsets er\-mög\-lichen, ohne den vorhandenen Quellcode zu ändern.
		}
	\end{itemize}

	\begin{enumerate}
		\item \t{
			What are the problems with the current implementation?
			List and briefly explain the issues.
		}{
			Welche Probleme hat die aktuelle Implementierung?
			Liste die Probleme auf und erkläre sie kurz.
		}		
		\begin{solution}
REDACTED
		\end{solution}
		
		\item \t{
			Which design pattern could solve the problems and why?
			Explain how the selected design pattern would support runtime switching of audio outputs, multiple simultaneous outputs, and the addition of new audio systems without changing existing code.
		}{
			Welches Design Pattern könnte die Probleme lösen und warum?
			Erkläre, wie das ausgewählte Design Pattern das Umschalten von Audio-Ausgaben zur Laufzeit, mehrere gleichzeitige Ausgaben und das Hinzufügen neuer Audio-Systeme ohne Änderung des bestehenden Codes unterstützen würde.
		}
		
		\begin{solution}
REDACTED
		\end{solution}
		
		\item \t{
			Refactor the provided Java code using the proposed design pattern.
			Provide the refactored Java-style code that satisfies all three requirements.
		}{
			Refaktoriere (= überarbeite) den bereitgestellten Java-Code unter Verwendung des vorgeschlagenen Design Patterns, so dass alle drei Anforderungen erfüllt werden.
		}
		\begin{solution}
REDACTED
		\end{solution}
	\end{enumerate}
}