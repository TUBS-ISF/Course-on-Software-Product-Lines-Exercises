\deftask{BonusFeatureIDESetup}
{\t{Set up FeatureIDE}{FeatureIDE einrichten}}{
	\t{
		We can employ \textsc{FeatureIDE} to model, implement and analyze product lines. 
		The aim of this task is to set up \textsc{FeatureIDE} and to address potential problems.
	}
	{
		\textsc{FeatureIDE} kann zum Modellieren, Implementieren, und Analysieren von kleinen Produktlinien eingesetzt werden.
		Hier soll es darum gehen, \textsc{FeatureIDE} das erste Mal einzurichten und eventuelle Probleme zu besprechen.
	}
	\begin{itemize}
		\item \t{
			\textbf{Prerequisite}: 
			Ensure that Java \footnote{\url{https://www.oracle.com/de/java/technologies/javase/jdk11-archive-downloads.html}} is installed on your computer. 
		}
		{
			Stelle sicher, dass Java\footnote{\url{https://www.oracle.com/de/java/technologies/javase/jdk11-archive-downloads.html}} auf deinem Computer installiert ist.
		}
		\item \t{
			Download the newest version of \textsc{FeatureIDE} from its homepage\footnote{\url{https://featureide.github.io/\#download}}, 
			this includes \emph{Eclipse with FeatureIDE, JDT, CDT, and AJDT}.
		}
		{
			Lade zunächst \textsc{FeatureIDE} in der neuesten Version von unserer Webseite\footnote{\url{https://featureide.github.io/\#download}} herunter (\emph{Eclipse mit FeatureIDE, JDT, CDT, and AJDT}).
		}
		\item \t{
			Unzip the zip-file (an installation is not necessary). 
			Start \textsc{FeatureIDE} by opening the file \texttt{eclipse/eclipse.exe} and confirm the choice of the default workspace
			 \texttt{../workspace} with \texttt{Launch}.
		}
		{
			Entpacke die heruntergeladene ZIP-Datei (eine Installation ist nicht notwendig).
			Starte \textsc{FeatureIDE} durch Öffnen der Datei \texttt{eclipse/eclipse.exe} und bestätige die Wahl des Standard-Workspace \texttt{../workspace} mit \texttt{Launch}.
		}
		\item \t{
			If you switch to the default workspace you can create new projects as normally or you can create a new FeatureIDE project, 
			which will be explained and used in the next practical task.
		}
	{
		Wenn du in den Standard-Workspace wechselst, 
		kannst du wie gewohnt neue Eclipse Projekte anlegen oder auch ein neues FeatureIDE-Projekt erzeugen.
	}
	\end{itemize}
}