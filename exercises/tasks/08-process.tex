\deftask{ProcessDomainApplicationEngineering}
{\t{Domain and Application Engineering}{Domain und Application Engineering}}{
	\t{
		Complete the following figure with the process steps employed in \emph{domain} and \emph{application engineering}.
		\begin{enumerate}
			\item Fill in each box with the name of the corresponding step (outer and inner boxes!).
			\item Add all missing edges / arrows.
			\item How and where can custom development be supported?
			\item Highlight which steps belong to the problem space and which steps belong to the solution space.
		\end{enumerate}
		Explain the resulting figure.
		In your opinion, in which activity should you invest the most efforts and why?
	}{
		Vervollständige die folgende Grafik mit den Prozessschritten von \emph{Domain Engineering} und \emph{Application Engineering}.
		\begin{enumerate}
			\item Fülle jeden Kasten mit dem Namen des entsprechenden Schritts (äußere und innere Boxen!).
			\item Ergänze alle fehlenden Kanten/Pfeile.
			\item Wie und wo kann kundenspezifische Entwicklung unterstützt werden?
			\item Markiere, welche Schritte zum \emph{Problem Space} und welche zum \emph{Solution Space} gehören.
		\end{enumerate}
		Erkläre die entstandene Abbildung.
		Wo sollte deiner Ansicht nach der meiste Aufwand einfließen und warum?
	}
	\begin{center}
		\resizebox{\linewidth}{!}{\input{../pics/spl-engineering.tex}}
	\end{center}
	\begin{enumerate}
		\item[(e)] \t{
			Besides, give examples for scenarios in which ...
		}{
			Nenne außerdem Beispielszenarien, in denen ...
		}
		\begin{enumerate}
			\item
			\t{almost all of the effort is invested in domain engineering}
			{fast der ganze Aufwand beim Domain Engineering liegt,}
			\item
			\t{almost all of the effort is invested in application engineering}
			{fast der ganze Aufwand beim Application Engineering liegt und}
			\item
			\t{the effort is balanced between domain and application engineering}
			{etwa gleich viel Aufwand beim Domain und Application Engineering liegt.}
		\end{enumerate}
		\begin{solution}
REDACTED
		\end{solution}
	\end{enumerate}
}

\deftask{ProcessAdoptionStrategies}
{\t{%
	Adoption Strategies for Software Product Lines
}{%
	Adoptionsstrategien für Software-Produktlinien
}}{
	\begin{enumerate}
	\item \t{
		Explain three typical adoption strategies for introducing software product lines.
	}{
		Erkläre drei typische Adoptionsstrategien zur Einführung von Software-Pro\-dukt\-linien.
	} 
	\begin{solution}
REDACTED
	\end{solution}

	\item \t{
		Recommend the most suitable strategy for each scenario given below and justify your answer.
	}{
		Empfehle die am besten geeignete Strategie für jedes der folgenden Szenarien und begründe deine Antwort.
	}
	\begin{enumerate}[label=\roman*.]
		\item \t{
			A company has developed three separate CRM systems for different regions.
			They share 60–70\% of features but were developed independently.
			The company wants to unify the codebase to reduce maintenance costs while keeping region-specific capabilities.
		}{
			Ein Unternehmen hat drei separate CRM-Systeme für verschiedene Regionen entwickelt.
			Sie teilen 60–70\% der Features, wurden aber unabhängig entwickelt.
			Das Unternehmen möchte die Codebasis vereinheitlichen, um Wartungskosten zu reduzieren und gleichzeitig regionsspezifische Funktionen zu erhalten.
		}
		\begin{solution}
REDACTED
		\end{solution}
		
		\item \t{
			A Software as a Service (SaaS) company started with a basic collaborative document editor.
			Over time, they've added optional features like chat, version control, and real-time co-editing.
			These are selectively enabled per customer.
		}{
			Ein Software-as-a-Service-Unternehmen begann mit einem einfachen kollaborativen Dokumenteneditor.
			Im Laufe der Zeit wurden optionale Features wie Chat, Versionskontrolle und Echtzeit-Co-Editing hinzugefügt.
			Diese werden pro Kunde selektiv aktiviert.
		}
		\begin{solution}
REDACTED
		\end{solution}
		
		\item \t{
			A national government is designing citizen's websites for different states.
			All websites must follow a strict federal specification with known features (e.g., tax filing, ID requests, appointments).
			No legacy software is reused.
		}{
			Eine nationale Regierung entwirft Bürger-Webseiten für verschiedene Bundesländer.
			Alle Webseiten müssen einer strikten föderalen Spezifikation mit bekannten Features folgen (z.B. Steuererklärung, Ausweisanträge, Terminvereinbarungen).
			Keine Legacy-Software wird wiederverwendet.
		}
		\begin{solution}
REDACTED
		\end{solution}
		
		\item \t{
			A small startup is planning a suite of finance apps: a budgeting tool, investment tracker, and tax calculator.
			All are being developed from scratch.
		}{
			Ein kleines Startup plant eine Suite von Finanz-Apps: ein Budgetierungs-Tool, einen Investment-Tracker und einen Steuerrechner.
			Alle werden von Grund auf entwickelt.
		}
		\begin{solution}
REDACTED
		\end{solution}
	\end{enumerate}
\end{enumerate}
}

\deftask{ProcessCostAnalysis}
{\t{%
	Cost Analysis for the Extractive Adoption Strategy
}{%
	Kostenanalyse bei der extraktiven Adoptionsstrategie
}}{
	\t{
		Explain the main message of the following figure:
	}{
		Erkläre die Kernaussagen der folgenden Abbildung.
	}
	\begin{center}
		\includegraphics[width=.7\linewidth,page=6,trim=2cm 17.5cm 11.5cm 3.8cm,clip]{../pics/papers/Krueger2016.pdf}
	\end{center}
	\t{
		In your answer refer to the following points:
	}{
		Gehe dabei auch auf folgende Aspekte ein:
	}
	\begin{enumerate}
		\item
		\t{
			When is it beneficial to introduce a software product line with the extractive adoption strategy?
		}{
			Wann lohnt es sich demnach, eine Software-Produktlinie nach dem extraktiven Vorgehen einzuführen?
		}
		\item
		\t{
			How do the fix investment cost $\Delta_f$ relate with the variable cost $\Delta_v$?
		}{
			Wie hängen die fixen Investitionskosten $\Delta_f$ und variablen Kosten $\Delta_v$ zusammen?
		}
		\item
		\t{
			Why are the scenarios I and II problematic?
		}{
			Warum sind die Szenarien I und II problematisch?
		}
		\item \t{
			Refer and explain in your answer the terms \emph{adoption barrier}, \emph{return-on-investment} and \emph{break-even-point}.
		}{
			Verwenden und erläutere in diesem Zusammenhang die Begriffe \emph{adoption barrier}, \emph{return-on-investment} und \emph{break-even-point}.
		}
	\end{enumerate}
	\t{
		To answer this task, you may read the marked sections of the paper attached below:
	}{
		Lies zur Bearbeitung ggf. die markierten Abschnitte des unten angehängten Papers:
	}

	\emph{Jacob Krüger, Wolfram Fenske, Jens Meinicke, Thomas Leich, and Gunter Saake. 2016.
	Extracting Software Product Lines: A Cost Estimation Perspective.
	In SPLC.
	ACM, NY, USA, 354--361.}
}

\deftask{ProcessScenarios}
{\t{%
	Choosing Implementation Techniques
}{%
	Auswahl von Implementierungstechniken
}}{
	\t{
		Which implementation technique(s) are most appropriate for the following product lines?
		Justify your answer based on the features and constraints provided.
	}{
		Welche Implementierungstechnik(en) sind für die folgenden Produktlinien am besten geeignet?
		Begründe deine Antwort basierend auf den angegebenen Features und Einschränkungen.
	}
	
	\begin{enumerate}
		\item \t{Mobile Banking App}{Mobile Banking App} \\
		\t{
			\textbf{Features:} biometric login, multi-currency support, custom themes, in-app chat
		}{
			\textbf{Features:} biometrischer Login, Mehrwährungsunterstützung, anpassbare Themes, In-App-Chat
		} \\
		\t{
			\textbf{Constraints:} security-critical, cross-platform (Android/iOS), regulatory audits require feature traceability, only needed features should be delivered
		}{
			\textbf{Einschränkungen:} sicherheits\-kritisch, plattformübergreifend (Android/iOS), be\-hörd\-liche Audits erfordern Feature Traceability, nur benötigte Features sollen ausgeliefert werden
		}
		\begin{solution}
REDACTED
		\end{solution}
		
		\item \t{Game Engine with Mod Support}{Spiel-Engine mit Mod-Unterstützung} \\
		\t{
			\textbf{Features:} physics engines, rendering backends, audio plugins, mod loader
		}{
			\textbf{Features:} Physik-Engines, Rendering-Backends, Audio-Plugins, Mod-Loader
		} \\
		\t{
			\textbf{Constraints:} third-party mods must be integrated without recompilation, runtime loading of user-selected features, real-time performance must be preserved
		}{
			\textbf{Einschränkungen:} Mods von Drittanbietern müssen ohne Neukompilierung integriert werden, Laden von Benutzer-gewählten Features zur Laufzeit, Echtzeit-Performance muss erhalten bleiben
		}
		\begin{solution}
REDACTED
		\end{solution}
		
		\item \t{Smart Car}{Smart Car} \\
		\t{
			\textbf{Features:} real-time sensor monitoring, GPS, Lidar, cameras, voice UI, driver assistance, over-the-air updates
		}{
			\textbf{Features:} Echtzeit-Sensor-Überwachung, GPS, Lidar, Kameras, Sprach-UI, Fahrassistenz, Over-the-Air-Updates
		} \\
		\t{
			\textbf{Constraints:} deployed on varied hardware platforms, strict real-time constraints, must pass ISO 26262 safety certification, must run on low-memory devices (64 KB RAM)
		}{
			\textbf{Einschränkungen:} auf verschiedenen Hardware-Plattformen eingesetzt, strik\-te Echtzeitanforderungen, muss ISO 26262 Sicherheitszertifizierung bestehen, muss auf Geräten mit wenig Speicher laufen (64 KB RAM)
		}
		\begin{solution}
REDACTED
		\end{solution}
		
		\item \t{Mental Health App}{Mental-Health-App} \\
		\t{
			\textbf{Features:} guided meditations, journaling, therapy chat, mood analytics
		}{
			\textbf{Features:} geführte Meditationen, Tagebuch, Therapie-Chat, Stimmungsanalyse
		} \\
		\t{
			\textbf{Constraints:} each user has a different license profile, must support offline use, features are toggled on/off per user
		}{
			\textbf{Einschränkungen:} jeder Benutzer hat ein anderes Lizenzprofil, muss Offline-Nutzung unterstützen, Features werden pro Benutzer aktiviert/deaktiviert
		}
		\begin{solution}
REDACTED
		\end{solution}
	\end{enumerate}
}

\deftask{ProcessChart}
{\t{%
	A Chart of Implementation Techniques
}{%
	Implementierungstechniken-Schaubild
}}{
	\t{
		Develop an infographic, diagram, or poster that summarizes and explains the most important relationships regarding the implementation techniques covered in this lecture.
		Which relationships and concepts you consider important and which presentation form you choose is up to you, as long as you can explain your graphic.
		You can limit yourself to the relationships discussed in the lecture and exercise, but you can also incorporate your own thoughts and ideas.
		For example, what conclusion you personally draw from this section of the lecture.
		Below you will find some suggestions that you can consider when designing your graphic (but don't have to).
	}{
		Entwickle eine Infografik bzw. ein Schaubild oder Poster, welche(s) die wichtigsten Zusammenhänge rund um die in dieser Vorlesung behandelten Implementierungstechniken zusammenfasst und erklärt.
		Welche Zusammenhänge und Begriffe du als wichtig erachtest und welche Darstellungsform du wählst, bleibt dir überlassen, so lange du deine Grafik erklären kannst.
		Du kannst dich auf die in der Vorlesung und Übung besprochenen Zusammenhänge beschränken, aber auch deine eigenen Gedanken und Ideen einfließen lassen.
		Zum Beispiel welches Fazit du persönlich aus diesem Abschnitt der Vorlesung ziehst.
		Im Folgenden findest du einige Anregungen, die du bei der Gestaltung deiner Grafik berücksichtigen kannst (aber nicht musst).
	}
	\begin{enumerate}
	\item \t{possible presentation forms:}{mögliche Darstellungsformen:}
		\begin{enumerate}
		\item \t{
			Concept/Mind Map, ontology/semantic network, organizational chart, flowchart, history/timeline, pictograms/illustrations, mosaic, \ldots
		}{
			Concept/Mind Map, Ontologie/semantisches Netz, Organigramm, Flussdiagramm, Historie/Zeitstrahl, Piktogramme/Illustrationen, Mosaik, \ldots
		}
		\end{enumerate}
	\item \t{possible relationships:}{mögliche Zusammenhänge:}
		\begin{enumerate}
		\item \t{
			\emph{Dimensions of variability} such as binding time, tool-/language-based, annotative/compositional, static/dynamic
		}{
			\emph{Dimensionen der Variabilität} wie Bindungszeitpunkt, tool-/sprachbasiert, annotativ/kompositional, statisch/dynamisch
		}
		\item \t{
			\emph{Quality criteria} such as feature traceability, separation of concerns, information hiding/cohesion/encapsulation, granularity, uniformity, obliviousness
		}{
			\emph{Qualitätskriterien} wie Feature Traceability, Separation of Concerns, Information Hiding/Kohäsion/Kapselung, Granularität, Uniformität, Obliviousness
		}
		\item \t{
			\emph{Problems} such as code tangling/scattering, code replication/cloning, preplanning, crosscutting concerns, fragile evolution
		}{
			\emph{Probleme} wie Code Tangling/Scattering, Code Replication/Cloning, Preplanning, querschneidende Belange, Fragile Evolution
		}
		\item \t{
			\emph{Scenarios} such as safety- or performance-critical systems, fine-grained extensions, adoption strategy, embedded systems, training effort, scalability
		}{
			\emph{Szenarien} wie sicherheits- oder performance-kritische Systeme, feingranulare Erweiterungen, Adoptionsstrategie, eingebettete Systeme, Schulungsaufwand, Skalierbarkeit
		}
		\item \t{and possibly more}{und ggf. weitere}
		\end{enumerate}
	\end{enumerate}
	\t{
		When in doubt, follow the Unix philosophy \emph{do one thing and do it well}.
		That is, don't try to explain \emph{all} relationships (which would probably lead to an overloaded and complex graphic), but rather some selected ones (which you then explain meaningfully and in sufficient detail and put into context).
		Tables are not admitted as a solution.
	}{
		Verfolge im Zweifelsfall die Unix-Philosophie \emph{do one thing and do it well}.
		Das heißt, versuche nicht, \emph{alle} Zusammenhänge zu erläutern (was vermutlich zu einer überladenen und komplexen Grafik führt), sondern einige ausgewählte (die du dann sinnvoll und hinreichend detailliert erklärst und in Bezug setzt).
		Tabellen sind als Lösung nicht zugelassen.
	}
}