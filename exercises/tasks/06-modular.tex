\deftask{ModularFundamentals}
{\t{Modular Design Principles}{Modulare Design-Prinzipien}}{
	\begin{enumerate}
		\item \t{
			Explain the terms \emph{modularity}, \emph{cohesion}, \emph{coupling}, and \emph{encapsulation}.
			How are they related?
		}{
			Erkläre die Begriffe \emph{Modularität}, \emph{Kohäsion}, \emph{Kopplung} und \emph{Kapselung}.
			Wie hängen sie zusammen?
		}
		\begin{solution}
REDACTED
		\end{solution}

		\item \t{
			Consider the following implementation.
		}{
			Betrachte die folgende Implementierung.
		}

		\myexample{\t{Change Implementation Example}{Beispiel einer Change-Implementierung}}{
			\includegraphics[scale=1.1]{code-java-tangled-render}
		}

		\begin{enumerate}
			\item[(i)] \t{
				Describe the code's cohesion and coupling characteristics.
			}{
				Beschreibe, inwiefern der Code Kohäsion und Kopplung aufweist.
			}
			\begin{solution}
REDACTED
			\end{solution}

			\item[(ii)] \t{
				Propose a modular redesign that addresses the problems identified above.
				In your solution, also explain why modularization is beneficial in this context.
				How could your design support product line variants such as \texttt{UndoSupport}?
			}{
				Schlage ein modulares Redesign vor, das die oben identifizierten Probleme adressiert.
				Erkläre in deiner Lösung auch, warum Modularisierung in diesem Kontext vorteilhaft ist.
				Wie könnte dein Design Produktlinien-Varianten wie \texttt{UndoSupport} unterstützen?
			}
			\begin{solution}
REDACTED
			\end{solution}
		\end{enumerate}
	\end{enumerate}
}

\deftask{ModularComponentGranularity}
{\t{Component Granularity and Library Scaling}{Komponentengranularität und Library Scaling}}{
	\t{
		You are developing a sensor-based embedded system that supports multiple hardware devices.
		Some sensors provide only temperature (in Celsius or Kelvin), others only air pressure (in Bar or Pascal), and a few support both.
		However, not all sensors support the same measurements or units.
		The current implementation combines everything into a single monolithic component:
	}{
		Du entwickelst ein sensorbasiertes eingebettetes System, das mehrere Hardware-Geräte unterstützt.
		Einige Sensoren liefern nur die Temperatur (in Celsius oder Kelvin), andere nur den Luftdruck (in Bar oder Pascal), und einige unterstützen beides.
		Allerdings unterstützen nicht alle Sensoren die gleichen Messgrößen oder Einheiten.
		Die aktuelle Implementierung kombiniert alles in einer einzigen monolithischen Komponente:
	}

	\myexample{\t{Monolithic Sensor Component}{Monolithische Sensor-Komponente}}{
		\includegraphics[scale=1.1]{code-java-monocomponent}
	}

	\begin{enumerate}
		\item \t{
			Based on this example, describe how would you design a modular sensor component suitable for a software product line.
			How can the design be adapted to handle different sensor capabilities and unit formats?
			Does \emph{glue code} play a role?
		}{
			Beschreibe basierend auf diesem Beispiel, wie du eine modulare Sensor-Komponente entwerfen würdest, die für eine Software-Produktlinie geeignet ist.
			Wie kann das Design angepasst werden, um verschiedene Sensor-Fähigkeiten und Einheitenformate zu handhaben?
			Spielt \emph{Glue Code} eine Rolle?
		}
		\begin{solution}
REDACTED
		\end{solution}

		\item \t{
			Explain the \textbf{library scaling problem}.
		}{
			Erkläre das \textbf{Library-Scaling-Problem}.
		}
		\begin{solution}
REDACTED
		\end{solution}
	\end{enumerate}
}

\deftask{ModularServiceComposition}
{\t{Service Composition}{Service-Komposition}}{
	\t{
		You are developing a product line for a vehicle tracking system with services like \texttt{TrackerService}, \texttt{AlertService}, and \texttt{ReportService}.
	}{
		Du entwickelst eine Produktlinie für ein Fahrzeug-Tracking-System mit Services wie \texttt{TrackerService}, \texttt{AlertService} und \texttt{ReportService}.
	}

	\myexample{\t{Microservices Example}{Microservices-Beispiel}}{
		\includegraphics[scale=1.1]{code-java-orchestration}
	}

	\begin{enumerate}
		\item \t{
			Which architectural approach is used in this design?
			What are the trade-offs of using this approach in a software product line?
		}{
			Welcher architektonische Ansatz wird in diesem Design verwendet?
			Was sind die Vor- und Nachteile der Verwendung dieses Ansatzes in einer Software-Produktlinie?
		}
		\begin{solution}
REDACTED
		\end{solution}

		\item \t{
			What is an alternative approach for service composition used in above code?
			How might it affect handling feature variability in a software product line?
		}{
			Was ist ein alternativer Ansatz für Service-Komposition im obigen Code?
			Wie könnte dies die Handhabung von Variabilität in einer Software-Produktlinie beeinflussen?
		}
		\begin{solution}
REDACTED
		\end{solution}
	\end{enumerate}
}

\deftask{ModularFramework}
{\t{Frameworks and Plugin Conflicts}{Frameworks und Plugin-Konflikte}}{
	\t{
		You are given a minimalistic plugin-based framework in Java.
	}{
		Gegeben ist ein minimalistisches Plugin-basiertes Framework in Java.
	}

	\myexample{\t{Minimalistic Dummy-Framework}{Minimalistisches Dummy-Framework}}{
		\includegraphics[scale=1.2]{code-framework}
	}

	\begin{enumerate}
		\item \t{
			Explain the term \emph{framework} by considering additionally the terms \emph{plugin/add-on}, \emph{interface}, and \emph{inversion of control}.
		}{
			Was versteht man unter einem Framework?
			Gehe dabei auch auf die Begriffe Plug-in/Add-on, Schnittstelle und \emph{Inversion of Control} ein.
		}
		\begin{solution}
REDACTED
		\end{solution}

		\item \t{
			Implement the following plugins:
		}{
			Implementiere die folgenden Plugins:
		}
		\begin{itemize}
			\item \t{
				A plugin that measures how long the application ran (approximately).
				The plugin should log the time so that it is visible to the user.
			}{
				Ein Plugin, das misst, wie lange die Anwendung lief (ungefähr).
				Das Plugin soll die Zeit so loggen, dass sie für den Benutzer sichtbar ist.
			}
			\item \t{
				A plugin that changes the color of the main button to \textcolor{red}{red}.
			}{
				Ein Plugin, das die Farbe des Hauptbuttons auf \textcolor{red}{Rot} ändert.
			}
			\item \t{
				A plugin that sets the application title to \texttt{"Hello World"}.
			}{
				Ein Plugin, das den Titel der Anwendung auf \texttt{"Hello World"} setzt.
			}
			\item \t{
				A plugin that changes the color of the main button to \textcolor{blue}{blue}.
			}{
				Ein Plugin, das die Farbe des Hauptbuttons auf \textcolor{blue}{Blau} ändert.
			}
		\end{itemize}
		\begin{solution}
REDACTED
		\end{solution}

		\item \t{
			Implement a dedicated \texttt{main} function that launches the framework with all your plugins.
			Do you observe any problems or unexpected behaviors?
			If so, how could they be addressed?
			Where does inversion of control occur in this design?
		}{
			Implementiere eine dedizierte \texttt{main}-Funktion, die das Framework mit all deinen Plugins startet.
			Beobachtest du irgendwelche Probleme oder unerwartetes Verhalten?
			Falls ja, wie könnten sie adressiert werden?
			Wo tritt Inversion of Control in diesem Design auf?
		}
		\begin{solution}
REDACTED
		\end{solution}

		\item \t{
			How would you extend the framework to support plugin priorities, so that potential conflicts between plugins can be resolved in a controlled way?
			How do such conflicts reflect the \textbf{preplanning problem}?
		}{
			Wie würdest du das Framework erweitern, um Plugin-Prioritäten zu unterstützen, sodass eventuelle Konflikte zwischen Plugins kontrolliert aufgelöst werden können?
			Wie spiegeln solche Konflikte im Kontext einer Produktlinie das \textbf{Preplanning-Problem} wider?
		}
		\begin{solution}
REDACTED
		\end{solution}
	\end{enumerate}
}

\deftask{ModularComparison}
{\t{%
	Comparison of Variability Implementation Techniques
}{%
	Vergleich von Implementierungstechniken für Variabilität
}}{
	\t{
		Compare the advantages, disadvantages, and use cases of implementation techniques discussed in the lecture so far.
		Use examples to support your arguments.
	}{
		Vergleiche die Vorteile, Nachteile und Anwendungsfälle der bisher in der Vorlesung besprochenen Implementierungstechniken.
		Verwende Beispiele zur Untermauerung deiner Argumente.
	}

	\begin{solution}
REDACTED
	\end{solution}
}
