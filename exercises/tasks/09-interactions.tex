\deftask{InteractionsFundamentals}
{\t{Feature Interactions: Fundamentals}{Grundlagen zu Feature-Interaktionen}}{
	\begin{enumerate}
		\item \t{
			Explain the term \emph{feature interaction} based on an example.
		}{
			Erläutere den Begriff der Feature-Interaktion an einem eigenen Beispiel.
		}
		\begin{solution}
REDACTED
		\end{solution}

		\item \t{
			Describe the meaning of a \emph{higher-order} feature interaction.
			Give an example of such an interaction.
		}{
			Was sind Feature-Interaktionen höherer Ordnung?
			Nenne ein Beispiel.
		}
		\begin{solution}
REDACTED
		\end{solution}

		\item \t{
			How many $t$-wise feature interactions are possible in a product line consisting of $n$ optional, unconstrained features?
			State formulas to compute this number for any $t$ and $t \in \{0,1,2,n\}$, specifically.
			How many feature interactions (across all $t$) are possible in total?
		}{
			Wie viele \emph{$t$-wise} Feature-Interaktionen kann eine Produktlinie mit $n$ optionalen, unab\-hän\-gigen Features maximal haben?
			Gib Formeln zur Berechnung dieser Zahl für beliebiges $t$ sowie konkret für $t \in \{0,1,2,n\}$ an.
			Wie viele Feature-Interaktionen (über alle $t$ hinweg) sind insgesamt möglich?
		}
		\begin{solution}
REDACTED
		\end{solution}
	\end{enumerate}
}

\deftask{InteractionsConsequences}
{\t{Consequences of Feature Interactions}{Konsequenzen von Feature-Interaktionen}}{
	\begin{enumerate}
		\item \t{
			Which (positive and negative) consequences can feature interactions have for software-product-line engineering?
		}{
			Welche (positiven und negativen) Konsequenzen können Feature-Interaktionen für die Entwicklung von Software-Produktlinien haben?
		}
		\begin{solution}
REDACTED
		\end{solution}

		\item \t{
			How can the number of expected feature interactions be reduced?
		}{
			Wie kann man die Anzahl der zu erwartenden Feature-Interaktionen reduzieren?
		}
		\begin{solution}
REDACTED
		\end{solution}

		\item \t{
			After this reduction, how can the remaining feature interactions be detected?
		}{
			Wie kann man die verbleibenden Feature-Interaktionen nach dieser Reduktion detektieren?
		}
		\begin{solution}
REDACTED
		\end{solution}
	\end{enumerate}
}

\deftask{InteractionsVariabilityBugDatabase}
{\t{Variability-Bug Analysis}{Analyse eines Variabilitäts-Bugs}}{
	\t{
		Explore the \emph{Variability Bug Database (VBDb)}\footnote{\url{https://vbdb.bitbucket.io/}} and select one real-world interaction fault (under ``Database'').
	}{
		Erkunde die \emph{Variability Bug Database (VBDb)}\footnote{\url{https://vbdb.bitbucket.io/}} und wähle einen echten Interaktionsfehler aus (unter ``Database'').
	}

	\begin{enumerate}
		\item \t{
			Explain the issue in your own words.
			Which features or options were involved?
		}{
			Erkläre das Problem in eigenen Worten.
			Welche Features oder Optionen waren beteiligt?
		}
		\begin{solution}
REDACTED
		\end{solution}

		\item \t{
			Classify the type of feature interaction (wanted/unwanted, static/dynamic).
			Is it only caused by the presence or absence of features, or both?
			Of which order $t$ is it?
		}{
			Klassifiziere die Art der Feature-Interaktion (gewollt/ungewollt, statisch/dynamisch).
			Ist sie nur durch das Vorhandensein oder Fehlen von Features verursacht, oder durch beides?
			Welcher Ordnung $t$ ist sie?
		}
		\begin{solution}
REDACTED
		\end{solution}

		\item \t{
			Analyze the bugfix provided in the VBDb.
			Does it apply one of the six strategies for handling feature interactions discussed in the lecture? If so, which one?
			Would you fix the issue differently? If so which strategy would you apply?
		}{
			Analysiere den in der VBDb bereitgestellten Bugfix.
			Wendet er eine der sechs in der Vorlesung diskutierten Strategien zum Umgang mit Feature-Interaktionen an? Wenn ja, welche?
			Würdest du das Problem anders beheben? Wenn ja, welche Strategie würdest du anwenden?
		}
		\begin{solution}
REDACTED
		\end{solution}
	\end{enumerate}
	\begin{solution}
REDACTED
	\end{solution}

}

\deftask{InteractionsStrategies}
{\t{Strategies for Resolving Feature Interactions}{Strategien zur Auflösung von Feature-Interaktionen}}{
	\t{
		Different strategies exist to handle feature interactions in software product lines, each with its own trade-offs.
		Compare any two strategies from the lecture by discussing their benefits, limitations, and appropriate scenarios for use.
		Illustrate your comparison using a concrete example.
	}{
		Es gibt verschiedene Strategien, um mit Feature-Interaktionen in Software-Produktlinien umzugehen, jede mit ihren eigenen Vor- und Nachteilen.
		Vergleiche zwei beliebige Strategien aus der Vorlesung, indem du ihre Vorteile, Einschränkungen und geeignete Anwendungsszenarien diskutierst.
		Veranschauliche deinen Vergleich anhand eines konkreten Beispiels.
	}

	\begin{solution}
REDACTED
	\end{solution}
}

\deftask{InteractionsDetection}
{\t{%
	Detecting Variability Bugs in Configurable Code
}{%
	Variability-Bugs in konfigurierbarem Code finden
}}{
	\t{
		Consider the following configurable Java code snippet.
		The system supports the following optional features (defined via preprocessor macros):
		\texttt{SECURE\_LOGIN}, \texttt{CACHING}, and \texttt{LOGGING}.
	}{
		Gegeben sei das folgende konfigurierbare Java-Code-Snippet.
		Das System unterstützt die folgenden optionalen Features (definiert über Präprozessor-Makros):
		\texttt{SECURE\_LOGIN}, \texttt{CACHING} und \texttt{LOGGING}.
	}

	\myexample{\t{%
		A configurable login handler
	}{%
		Ein konfigurierbarer Login-Handler
	}}{
		\includegraphics[width=1.2\linewidth]{code-feature-interaction-debug}
	}

	\begin{enumerate}
		\item \t{
			Identify configurations in which the above code does not compile, as well as configurations in which the code compiles but fails at runtime.
			What causes these issues?
		}{
			Identifiziere Konfigurationen, in denen der obige Code nicht kompiliert, sowie Konfigurationen, in denen der obige Code kompiliert, aber zur Laufzeit fehlschlägt.
			Was verursacht diese Probleme?
		}

		\item \t{
			Are these static or dynamic feature interactions?
			Of which order $t$ are they?
			Are these ``real'' feature interactions (tied to the inherent complexity of the interacting features)?
		}{
			Sind dies statische oder dynamische Feature-Interaktionen?
			Welcher Ordnung $t$ sind sie?
			Sind dies ``echte'' Feature-Interaktionen (bedingt durch die inhärente Komplexität der interagierenden Features)?
		}

		\item \t{
			Would testing with the default configuration (\texttt{SECURE\_LOGIN}, \texttt{CACHING}, and \texttt{LOGGING}) have caught these issues?
			Suppose you did not identify these interactions yet -- which other configurations would you also test to increase the chances of detecting bugs?
		}{
			Hätte das Testen mit der Standard-Konfiguration (\texttt{SECURE\_LOGIN}, \texttt{CACHING} und \texttt{LOGGING}) diese Probleme erkannt?
			Angenommen, du hättest diese Interaktionen noch nicht identifiziert – welche anderen Konfigurationen würdest du noch testen, um die Chancen zur Fehlererkennung zu erhöhen?
		}
	\end{enumerate}

	\begin{solution}
REDACTED
	\end{solution}
}