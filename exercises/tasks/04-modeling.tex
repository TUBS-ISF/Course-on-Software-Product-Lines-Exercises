\deftask{ModelingFundamentals}
{\t{Feature Modeling Fundamentals}{Grundlagen zu Feature-Modellen}}{
	\t{
		Consider the following feature model for a mobile app product line.
		(Abbreviations of the features names are highlighted in red.)
	}{
		Gegeben sei folgendes Feature-Modell für eine Mobile-App-Produktlinie.
		(Abkürzungen der Feature-Namen sind rot hervorgehoben.)
	}

	\myexample{\t{%
		Feature Model: Smart Mobile App
	}{%
		Feature-Modell: Smarte Mobile App
	}}{
		\centering\sffamily
		\featureDiagram{
			\textbf{\textcolor{red}{M}}obile\textbf{\textcolor{red}{A}}pp,abstract
			[\textbf{\textcolor{red}{L}}ogin,mandatory,abstract
			[\textbf{\textcolor{red}{E}}mail,alternative,concrete]
			[\textbf{\textcolor{red}{G}}oogle,concrete]
			[\textbf{\textcolor{red}{F}}acebook,concrete]]
			[\textbf{\textcolor{red}{N}}otifications,optional,abstract
			[\textbf{\textcolor{red}{P}}ush,optional,concrete]
			[\textbf{\textcolor{red}{I}}n\textbf{\textcolor{red}{A}}pp,optional,concrete]]
			[{\textbf{\textcolor{red}{P}}aymen\textbf{\textcolor{red}{t}}},optional,abstract
			[\textbf{\textcolor{red}{C}}redit\textbf{\textcolor{red}{C}}ard,or,concrete]
			[\textbf{\textcolor{red}{P}}ay\textbf{\textcolor{red}{P}}a\textbf{\textcolor{red}{l}},concrete]
			[\textbf{\textcolor{red}{V}}oucher,concrete]]
			[{\textbf{\textcolor{red}{A}}nalytics},optional,concrete]
			}
			$\text{\textbf{\textcolor{red}{P}}ush} \implies \text{\textbf{\textcolor{red}{G}}oogle}$ \\
			$\text{\textbf{\textcolor{red}{A}}nalytics} \implies \text{\textbf{\textcolor{red}{C}}redit\textbf{\textcolor{red}{C}}ard}$ \\
			$\text{\textbf{\textcolor{red}{P}}ay\textbf{\textcolor{red}{P}}a\textbf{\textcolor{red}{l}}} \implies \text{\textbf{\textcolor{red}{I}}n\textbf{\textcolor{red}{A}}pp}$
	}

	\begin{enumerate}
		\item
		\t{
			What is a feature model, and what are its typical purposes and applications in software product line engineering?
		}{
			Was ist ein Feature-Modell und was sind seine typischen Zwecke und Anwendungen im Software-Produktlinien-Engineering?
		}

		\begin{solution}
REDACTED
		\end{solution}

		\item
		\t{
			Explain the given feature model, including its structure and constraints.
			Provide one valid and one invalid configuration along with the reasoning behind each choice.
		}{
			Erkläre das gegebene Feature-Modell, einschließlich seiner Struktur und Constraints.
			Gib eine gültige und eine ungültige Konfiguration an und begründe deine Wahl.
		}

		\begin{solution}
REDACTED
		\end{solution}
	\end{enumerate}
}

\deftask{ModelingAnalysis}
{\t{Feature Model Analysis}{Feature-Modell-Analyse}}{
	\t{
		Analyze the following feature model for a simple music player device.
	}{
		Analysiere das folgende Feature-Modell für ein einfaches Musikabspielgerät.
	}
	\myexample{\t{%
		Feature Model: Simple Music Player
	}{%
		Feature-Modell: Einfacher Musikplayer
	}}{
		\centering\sffamily
		\featureDiagram{
			MusicPlayer,abstract
			[Playback,mandatory,abstract
			[MP3,alternative,concrete]
			[FLAC,concrete]]
			[Connectivity,optional,abstract
			[Bluetooth,optional,concrete]
			[WiFi,optional,concrete]]
			[Storage,optional,abstract
			[SDCard,optional,concrete]
			[InternalMemory,optional,concrete]]
		}

		$\text{WiFi} \pimplies \text{InternalMemory}$ \\
		$\neg \text{FLAC}$ \\
		$\text{Bluetooth} \pimplies \text{InternalMemory}$ \\
		$\text{WiFi} \pimplies \neg \text{SDCard}$
	}
	\begin{enumerate}
		\item \t{
			Is the feature model void? Justify your answer.
		}{
			Ist das Feature-Modell inkonsistent?
			Begründe deine Antwort.
		}
		\begin{solution}
REDACTED
		\end{solution}

		\item \t{
			Identify which features are core and which are dead.
		}{
			Identifiziere, welche Features \emph{core} (nicht abwählbar) und welche tot sind.
		}
		\begin{solution}
REDACTED
		\end{solution}

		\item \t{
			Propose a new constraint that would make Bluetooth a dead feature.
		}{
			Schlage eine neue Bedingung vor, die Bluetooth zu einem toten Feature macht.
		}
		\begin{solution}
REDACTED
		\end{solution}

		\item \t{
			How can a SAT solver be applied to perform the analyses (a) and (b)?
		}{
			Wie kann ein SAT-Solver angewendet werden, um die Analysen (a) und (b) durchzuführen?
		}
		\begin{solution}
REDACTED
		\end{solution}
	\end{enumerate}
}

\deftask{ModelingTransformation}
{\t{Feature Model Transformation}{Feature-Modell-Transformation}}{
	\t{
		You are given the following feature model.
	}{
		Gegeben sei folgendes Feature-Modell.
	}
	\myexample{Graph Product Line}{
		\centering\sffamily
		\featureDiagram{
			Graph,abstract
			[Edges,mandatory,abstract
			[Directed,alternative,concrete]
			[Undirected,concrete]]
			[Algorithms,optional,abstract,
			[DFS,or,concrete]
			[CycleDetector,concrete]]
		}

		$\text{CycleDetector} \pimplies \text{Directed}$
	}
	\begin{enumerate}
		\item \t{
			How many valid products can be derived from the given feature model?
		}{
			Wie viele Produkte beschreibt das obige Feature-Modell?
		}
		\item \t{
			How can you in general determine the number of configurable products?
			What is the difficulty?
		}{
			Wie kann man beim Bestimmen dieser Anzahl allgemein vorgehen (und wo liegt die Schwierigkeit)?
		}
		\begin{solution}
REDACTED
		\end{solution}
		\item
		\t{
			Diagrams are not the only language for feature models.
			Transform the given diagram into a
			\begin{itemize}
				\item propositional formula,
				\item CNF (use the propositional formula you just created),
				\item set of all valid configurations (subset of the superset of all features).
			\end{itemize}
			What are the advantages and disadvantages of these notations?
			Do you know other languages for constraints?
		}{
			Es gibt neben der Diagrammschreibweise noch andere Arten, Feature-Modelle darzustellen.
			Überführe das obige Modell (1) in eine aussagenlogische Formel und (2) in eine Menge aller gültigen Konfigurationen (Teilmenge der Potenzmenge aller Features).
			Welche Vor- und Nachteile haben diese Darstellungen?
			Kennst du andere Sprachen für Constraints?
		}
		\begin{solution}
REDACTED
		\end{solution}
	\end{enumerate}
}

% for this task, we have different variants, which can be enabled manually by uncommenting the respective lines below
\deftask{ModelingDerivation}
{\t{Derive Feature Models from Requirements}{Feature-Modelle aus Anforderungen ableiten}}{
	\t{
		Often, customers may already have various requirements in mind, but they are not formalized as feature models.
		In the following scenarios, help the customer to specify their requirements.
	}{
		Oft haben Kunden bereits verschiedene Anforderungen im Kopf, aber diese sind nicht als Feature-Modelle formalisiert.
		In den folgenden Szenarien hilfst du dem Kunden, seine Anforderungen zu spezifizieren.
	}

	\begin{enumerate}
		\item
		\t{
			Our first customer wants to build a product line for
			smart home control devices.
			% snack vending machines.
			They mention the following requirements:
		}{
			Unser erster Kunde möchte eine Produktlinie für
			Smart-Home-Steuergeräte
			% Snackautomaten
			aufbauen.
			Er nennt die folgenden Anforderungen:
		}
		\begin{itemize}
			\item \t{
				Voice control requires a microphone and a speaker.
			}{
				Sprachsteuerung erfordert ein Mikrofon und einen Lautsprecher.
			}
			\item \t{
				Remote control via smartphone requires Wi-Fi.
			}{
				Fernsteuerung über Smartphone erfordert WLAN.
			}
			\item \t{
				Displaying a user interface requires a touchscreen.
			}{
				Die Anzeige einer Benutzeroberfläche erfordert einen Touchscreen.
			}
			\item \t{
				Motion detection requires an infrared sensor.
			}{
				Bewegungserkennung erfordert einen Infrarotsensor.
			}
			\item \t{
				Outdoor installation requires weatherproofing.
			}{
				Außeninstallation erfordert Wetterfestigkeit.
			}
		\end{itemize}
		% \begin{itemize}
		% 	\item \t{
		% 		Paying with card requires a card reader and a pin pad.
		% 	}{
		% 		Kartenzahlung erfordert einen Kartenleser und ein Pin-Pad.
		% 	}
		% 	\item \t{
		% 		Selling ice cream requires a freezer.
		% 	}{
		% 		Der Verkauf von Eiscreme erfordert eine Gefriereinheit.
		% 	}
		% 	\item \t{
		% 		For airports, multiple language support is needed.
		% 	}{
		% 		Für Flughäfen ist eine mehrsprachige Unterstützung erforderlich.
		% 	}
		% 	\item \t{
		% 		The option to pay with cash requires further security measures.
		% 	}{
		% 		Die Option der Barzahlung erfordert zusätzliche Sicherheitsmaßnahmen.
		% 	}
		% 	\item \t{
		% 		Selling bottled drinks requires a transport system.
		% 	}{
		% 		Der Verkauf von Flaschengetränken erfordert ein Transportsystem.
		% 	}
		% \end{itemize}
		\t{
			Create a feature model representing these requirements.
			What dependencies or features would you suggest adding?
		}{
			Erstelle ein Feature-Modell, das diese Anforderungen abbildet.
			Fehlen Abhängigkeiten oder Features?
		}
		\begin{solution}
REDACTED
		\end{solution}
		\vspace{1\baselineskip}
		\item
		\t{
			Our second customer wants to
			build a product line for smartphones.
			% turn their family of bikes into a product line.
			They are unsure about dependencies, but provide examples of their sold models:
		}{
			Unser zweiter Kunde möchte
			eine Produktlinie für Smartphones aufbauen.
			% seine Fahrradfamilie in eine Produktlinie umwandeln.
			Er ist sich über Abhängigkeiten unsicher, gibt jedoch Beispiele seiner verkauften Modelle an:
		}
		\begin{itemize}
			\item \t{
				5.5 inch Display, Front Camera, Fingerprint Sensor, NFC, Wi-Fi
			}{
				5,5-Zoll-Display, Frontkamera, Fingerabdrucksensor, NFC, WLAN
			}
			\item \t{
				6.1 inch Display, Front Camera, Back Camera, NFC, Wi-Fi, 5G
			}{
				6,1-Zoll-Display, Frontkamera, Rückkamera, NFC, WLAN, 5G
			}
			\item \t{
				5.0 inch Display, Front Camera, Fingerprint Sensor, Wi-Fi
			}{
				5,0-Zoll-Display, Frontkamera, Fingerabdrucksensor, WLAN
			}
			\item \t{
				6.5 inch Display, Back Camera, Wi-Fi, 5G
			}{
				6,5-Zoll-Display, Rückkamera, WLAN, 5G
			}
		\end{itemize}
		% \begin{itemize}
		% 	\item \t{
		% 		\{24 inches, Front light, Back light, Dynamo, Brakes, Bell\}
		% 	}{
		% 		\{24 Zoll, Vorderlicht, Rücklicht, Dynamo, Bremsen, Klingel\}
		% 	}
		% 	\item \t{
		% 		\{18 inches, Front light, Back light, Dynamo, Brakes, Jockey wheel\}
		% 	}{
		% 		\{18 Zoll, Vorderlicht, Rücklicht, Dynamo, Bremsen, Stützrad\}
		% 	}
		% 	\item \t{
		% 		\{16 inches, Brakes, Jockey wheel, Bell\}
		% 	}{
		% 		\{16 Zoll, Bremsen, Stützrad, Klingel\}
		% 	}
		% 	\item \t{
		% 		\{26 inches, Front light, Dynamo, Brakes\}
		% 	}{
		% 		\{26 Zoll, Vorderlicht, Dynamo, Bremsen\}
		% 	}
		% 	\item \t{
		% 		\{22 inches, Back light, Dynamo, Brakes\}
		% 	}{
		% 		\{22 Zoll, Rücklicht, Dynamo, Bremsen\}
		% 	}
		% 	\item \t{
		% 		\{16 inches, Front light, Back light, Dynamo, Brakes\}
		% 	}{
		% 		\{16 Zoll, Vorderlicht, Rücklicht, Dynamo, Bremsen\}
		% 	}
		% \end{itemize}
		\t{
			Create a feature model based on these sample products.
			What difficulties might arise when creating a feature model from only a small set of example products?
		}{
			Erstelle ein Feature-Modell basierend auf diesen Beispielprodukten.
			Welche Schwierigkeiten könnten auftreten, wenn man ein Feature-Modell nur aus einer kleinen Auswahl von Produkten erstellt?
		}
		\begin{solution}
REDACTED
		\end{solution}
	\end{enumerate}
}

% for this task, we have different variants, which can be enabled manually by uncommenting the respective lines below
\deftask{ModelingCountSubwaySandwiches}
{\t{How many sandwiches does Subway sell?}{Wie viele Sandwiches gibt es bei Subway?}}{
	\begin{enumerate}
	\item \t{
		Try to calculate (using a method of your choice) the (approximate) number of all sandwiches that can be ordered at Subway.
		Use the given order form.
		Explain your idea.
		Is it manual or machine-supported?

		Can you find a solution?
		If you have difficulties, document them.
		How well does your result approximate the actual solution?
	}{
		Versuche, mit einer Methode deiner Wahl (eine Näherung für) die Anzahl aller Sandwiches zu berechnen, die man bei Subway bestellen kann.
		Nutze dafür das angegebene Bestellformular.
		Erläutere deinen Ansatz.
		Ist dieser manuell oder maschinell unterstützt?

		Gelingt es dir, eine Lösung zu finden?
		Falls Schwierigkeiten auftreten, dokumentiere diese.
		Was glaubst du, wie gut ist deine Näherung?
	}
	\begin{solution}
REDACTED
	\end{solution}
	\item \t{
		Are there aspects regarding this form you would criticize as a product-line engineer?
		How do these aspects make finding the right answer more difficult?
	}{
		Gibt es Aspekte, die du als Produktlinieningenieur an dem Bestellformular kritisieren würdest (und wenn ja, welche)?
		Inwiefern erschweren diese Aspekte das Finden einer genauen Lösung?
	}
	\begin{solution}
REDACTED
	\end{solution}
	\end{enumerate}

	\vspace*{1ex}
	\hspace*{-.1\linewidth}
	\includegraphics[width=1.2\linewidth,page=3,trim=0.5cm 2cm 0.8cm 0.5cm,clip]{../pics/subway-sandwiches-form.pdf}
	% \footnotesize
	% \t{
	% 	If you decide to use a machine-supported solution for this task, here are some hints:
	% }{
	% 	Solltest du dich für eine maschinell unterstützte Lösung der Aufgabe entscheiden, hier einige Hinweise:
	% }
	% \begin{itemize}
	% 	\item \t{
	% 		FeatureIDE can count how many configurations a feature model represents.
	% 		You can find this information in the \emph{FeatureIDE Statistics} view or, based on a partial configuration, in the \emph{Configuration Editor}.
	% 	}{
	% 		FeatureIDE kann zählen, wie viele Konfigurationen ein Feature-Modell repräsentiert.
	% 		Diese Informationen findest du im \emph{FeatureIDE Statistics} View oder, basierend auf einer Teilkonfiguration, im \emph{Configuration Editor}.
	% 	}
	% 	\item \t{
	% 		Counting large configuration spaces with FeatureIDE can be difficult.
	% 		If you encounter difficulties, use a \#SAT solver (research briefly what the difference to a SAT solver is) like \emph{d4}.\footnote{\url{https://github.com/ekuiter/torte/raw/refs/heads/main/src/docker/solver/emse-2023/d4}}
	% 		These are Linux binaries, so under Windows consider using a virtual machine or WSL (and \texttt{chmod +x countAntom} to make the file executable).
	% 		You can run it with \texttt{./countAntom <DIMACS file>} or \texttt{./d4 <DIMACS file>}.
	% 	}{
	% 		Das Zählen von großen Konfigurationsräumen mit FeatureIDE kann unter Umständen schwierig sein.
	% 		Falls du dabei Schwierigkeiten hast, verwende einen \#SAT-Solver (recherchiere in dem Fall kurz, was der Unterschied zu einem SAT-Solver ist) wie \emph{d4}.\footnote{\url{https://github.com/ekuiter/torte/raw/refs/heads/main/src/docker/solver/emse-2023/d4}}
	% 		Das sind Linux-Binaries, also verwende unter Windows ggf. eine virtuelle Maschine oder WSL (und \texttt{chmod +x countAntom}, um die Datei ausführbar zu machen).
	% 		Der Aufruf erfolgt mit \texttt{./countAntom <DIMACS file>} oder \texttt{./d4 <DIMACS file>}.
	% 	}
	% 	\item \t{
	% 		You can export a DIMACS file from FeatureIDE by right-clicking on \texttt{model.xml > FeatureIDE > Export Feature Model} (research briefly what this data format represents).
	% 	}{
	% 		Ein DIMACS-File kansst du in FeatureIDE mit Rechtsklick auf \texttt{model.xml > FeatureIDE > Export Feature Model} exportieren (recherchiere kurz, was dieses Datenformat repräsentiert).
	% 	}
	% 	\item \t{
	% 		On Windows, note that this DIMACS file uses \emph{LF} and not \emph{CRLF} as line ending.
	% 		This can be set in common text editors.
	% 	}{
	% 		Unter Windows ist zu beachen, dass dieses DIMACS-File \emph{LF} und nicht \emph{CRLF} als Zeilenende verwendet.
	% 		Dies kann in gängigen Texteditoren eingestellt werden.
	% 	}
	% \end{itemize}
}
