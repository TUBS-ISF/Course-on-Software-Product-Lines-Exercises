\documentclass{standalone}
\usepackage{xcolor}
\usepackage{listings}
\input{../Course-on-Software-Product-Lines/slides/template/colorscheme}

\lstdefinelanguage{MyC}{
	language=C,
	morekeywords={if, else, for, while, return, void, int, float, double},
	morekeywords=[2]{define, include},
	morekeywords=[3]{ifdef, ifndef, endif, else, elif},
}

\lstdefinestyle{modernc}{
	language=MyC,
	tabsize=2,
	breaklines=true,
	basicstyle=\footnotesize\ttfamily,
	commentstyle=\color{green!50!black}\ttfamily,
	stringstyle=\color{red}\ttfamily,
	showstringspaces=false,
	columns=fullflexible,
	keywordstyle=[2]\color{violet}\ttfamily, % define, include
	keywordstyle=[3]\color{blue}\ttfamily,   % ifdef etc.
	keywordstyle=\color{blue}\ttfamily,
	literate={%
		\#}{{\textcolor{blue}{\#}}}1  % Color the `#` symbol in teal to match preprocessor directives
	{else}{{\textcolor{blue}{else}}}4   % Ensure that #else is also teal
	}

\begin{document}
	\begin{lstlisting}[style=modernc]
#ifdef PRECISION_HIGH
    #define p_t double
    #define EPSILON 0.000001
    #ifdef UNIT_KELVIN
        #define CONVERT(x) ((x) * 0.1 + 273.15)
    #else
        #define CONVERT(x) ((x) * 0.1)
    #endif
#else
    #define p_t float
    #define EPSILON 0.001f
    #define CONVERT(x) (x)
#endif

// Thermal mode and sensor macros
#ifdef MODE_THERMAL
    #define SENSOR_COUNT 3
    #ifdef SENSOR_ADVANCED
        #define READ_RAW(i) (300 + (i) * 10)
    #else
        #define READ_RAW(i) (200 + (i) * 8)
    #endif
#endif

void readSensor() {
#ifdef MODE_THERMAL
    for (int i = 0; i < SENSOR_COUNT; ++i) {
        p_t value = CONVERT(READ_RAW(i));
        if (value > EPSILON) printf("Sensor[%d] active: %.2f\n", i, value);
    }
#else
    printf("Thermal mode disabled.\n");
#endif
}

	\end{lstlisting}
\end{document}
